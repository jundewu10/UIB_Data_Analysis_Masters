% Options for packages loaded elsewhere
\PassOptionsToPackage{unicode}{hyperref}
\PassOptionsToPackage{hyphens}{url}
\PassOptionsToPackage{dvipsnames,svgnames*,x11names*}{xcolor}
%
\documentclass[
]{article}
\usepackage{lmodern}
\usepackage{amssymb,amsmath}
\usepackage{ifxetex,ifluatex}
\ifnum 0\ifxetex 1\fi\ifluatex 1\fi=0 % if pdftex
  \usepackage[T1]{fontenc}
  \usepackage[utf8]{inputenc}
  \usepackage{textcomp} % provide euro and other symbols
\else % if luatex or xetex
  \usepackage{unicode-math}
  \defaultfontfeatures{Scale=MatchLowercase}
  \defaultfontfeatures[\rmfamily]{Ligatures=TeX,Scale=1}
\fi
% Use upquote if available, for straight quotes in verbatim environments
\IfFileExists{upquote.sty}{\usepackage{upquote}}{}
\IfFileExists{microtype.sty}{% use microtype if available
  \usepackage[]{microtype}
  \UseMicrotypeSet[protrusion]{basicmath} % disable protrusion for tt fonts
}{}
\makeatletter
\@ifundefined{KOMAClassName}{% if non-KOMA class
  \IfFileExists{parskip.sty}{%
    \usepackage{parskip}
  }{% else
    \setlength{\parindent}{0pt}
    \setlength{\parskip}{6pt plus 2pt minus 1pt}}
}{% if KOMA class
  \KOMAoptions{parskip=half}}
\makeatother
\usepackage{xcolor}
\IfFileExists{xurl.sty}{\usepackage{xurl}}{} % add URL line breaks if available
\IfFileExists{bookmark.sty}{\usepackage{bookmark}}{\usepackage{hyperref}}
\hypersetup{
  pdftitle={SVM: Práctica de regresión},
  pdfauthor={Jun De Wu},
  colorlinks=true,
  linkcolor=red,
  filecolor=Maroon,
  citecolor=purple,
  urlcolor=blue,
  pdfcreator={LaTeX via pandoc}}
\urlstyle{same} % disable monospaced font for URLs
\usepackage[margin=1in]{geometry}
\usepackage{color}
\usepackage{fancyvrb}
\newcommand{\VerbBar}{|}
\newcommand{\VERB}{\Verb[commandchars=\\\{\}]}
\DefineVerbatimEnvironment{Highlighting}{Verbatim}{commandchars=\\\{\}}
% Add ',fontsize=\small' for more characters per line
\usepackage{framed}
\definecolor{shadecolor}{RGB}{248,248,248}
\newenvironment{Shaded}{\begin{snugshade}}{\end{snugshade}}
\newcommand{\AlertTok}[1]{\textcolor[rgb]{0.94,0.16,0.16}{#1}}
\newcommand{\AnnotationTok}[1]{\textcolor[rgb]{0.56,0.35,0.01}{\textbf{\textit{#1}}}}
\newcommand{\AttributeTok}[1]{\textcolor[rgb]{0.77,0.63,0.00}{#1}}
\newcommand{\BaseNTok}[1]{\textcolor[rgb]{0.00,0.00,0.81}{#1}}
\newcommand{\BuiltInTok}[1]{#1}
\newcommand{\CharTok}[1]{\textcolor[rgb]{0.31,0.60,0.02}{#1}}
\newcommand{\CommentTok}[1]{\textcolor[rgb]{0.56,0.35,0.01}{\textit{#1}}}
\newcommand{\CommentVarTok}[1]{\textcolor[rgb]{0.56,0.35,0.01}{\textbf{\textit{#1}}}}
\newcommand{\ConstantTok}[1]{\textcolor[rgb]{0.00,0.00,0.00}{#1}}
\newcommand{\ControlFlowTok}[1]{\textcolor[rgb]{0.13,0.29,0.53}{\textbf{#1}}}
\newcommand{\DataTypeTok}[1]{\textcolor[rgb]{0.13,0.29,0.53}{#1}}
\newcommand{\DecValTok}[1]{\textcolor[rgb]{0.00,0.00,0.81}{#1}}
\newcommand{\DocumentationTok}[1]{\textcolor[rgb]{0.56,0.35,0.01}{\textbf{\textit{#1}}}}
\newcommand{\ErrorTok}[1]{\textcolor[rgb]{0.64,0.00,0.00}{\textbf{#1}}}
\newcommand{\ExtensionTok}[1]{#1}
\newcommand{\FloatTok}[1]{\textcolor[rgb]{0.00,0.00,0.81}{#1}}
\newcommand{\FunctionTok}[1]{\textcolor[rgb]{0.00,0.00,0.00}{#1}}
\newcommand{\ImportTok}[1]{#1}
\newcommand{\InformationTok}[1]{\textcolor[rgb]{0.56,0.35,0.01}{\textbf{\textit{#1}}}}
\newcommand{\KeywordTok}[1]{\textcolor[rgb]{0.13,0.29,0.53}{\textbf{#1}}}
\newcommand{\NormalTok}[1]{#1}
\newcommand{\OperatorTok}[1]{\textcolor[rgb]{0.81,0.36,0.00}{\textbf{#1}}}
\newcommand{\OtherTok}[1]{\textcolor[rgb]{0.56,0.35,0.01}{#1}}
\newcommand{\PreprocessorTok}[1]{\textcolor[rgb]{0.56,0.35,0.01}{\textit{#1}}}
\newcommand{\RegionMarkerTok}[1]{#1}
\newcommand{\SpecialCharTok}[1]{\textcolor[rgb]{0.00,0.00,0.00}{#1}}
\newcommand{\SpecialStringTok}[1]{\textcolor[rgb]{0.31,0.60,0.02}{#1}}
\newcommand{\StringTok}[1]{\textcolor[rgb]{0.31,0.60,0.02}{#1}}
\newcommand{\VariableTok}[1]{\textcolor[rgb]{0.00,0.00,0.00}{#1}}
\newcommand{\VerbatimStringTok}[1]{\textcolor[rgb]{0.31,0.60,0.02}{#1}}
\newcommand{\WarningTok}[1]{\textcolor[rgb]{0.56,0.35,0.01}{\textbf{\textit{#1}}}}
\usepackage{graphicx,grffile}
\makeatletter
\def\maxwidth{\ifdim\Gin@nat@width>\linewidth\linewidth\else\Gin@nat@width\fi}
\def\maxheight{\ifdim\Gin@nat@height>\textheight\textheight\else\Gin@nat@height\fi}
\makeatother
% Scale images if necessary, so that they will not overflow the page
% margins by default, and it is still possible to overwrite the defaults
% using explicit options in \includegraphics[width, height, ...]{}
\setkeys{Gin}{width=\maxwidth,height=\maxheight,keepaspectratio}
% Set default figure placement to htbp
\makeatletter
\def\fps@figure{htbp}
\makeatother
\setlength{\emergencystretch}{3em} % prevent overfull lines
\providecommand{\tightlist}{%
  \setlength{\itemsep}{0pt}\setlength{\parskip}{0pt}}
\setcounter{secnumdepth}{5}
\renewcommand{\contentsname}{Contenidos}

\title{SVM: Práctica de regresión}
\author{Jun De Wu}
\date{12/06/2021}

\begin{document}
\maketitle

{
\hypersetup{linkcolor=blue}
\setcounter{tocdepth}{2}
\tableofcontents
}
\begin{Shaded}
\begin{Highlighting}[]
\KeywordTok{require}\NormalTok{(tidyverse)}
\KeywordTok{require}\NormalTok{(e1071)}
\KeywordTok{require}\NormalTok{(MLmetrics)}
\KeywordTok{require}\NormalTok{(caTools)}
\KeywordTok{require}\NormalTok{(ggcorrplot)}
\KeywordTok{require}\NormalTok{(FNN)}
\KeywordTok{require}\NormalTok{(rpart)}
\KeywordTok{require}\NormalTok{(rpart.plot)}
\KeywordTok{require}\NormalTok{(randomForest)}
\KeywordTok{require}\NormalTok{(neuralnet)}
\end{Highlighting}
\end{Shaded}

\hypertarget{introducciuxf3n}{%
\section{Introducción}\label{introducciuxf3n}}

Nuestro conjunto de datos contiene información sobre los bloques de
vivienda de California en el año 1990. De media, un bloque de vivienda
contenía 1425.5 personas que vivían en un espacio reducido. En total
tenemos 20640 observaciones y 9 variables.

\hypertarget{exploraciuxf3n-de-los-datos}{%
\section{Exploración de los datos}\label{exploraciuxf3n-de-los-datos}}

\begin{Shaded}
\begin{Highlighting}[]
\NormalTok{dataset <-}\StringTok{ }\KeywordTok{read.csv}\NormalTok{(}\StringTok{"cadata2.csv"}\NormalTok{, }\DataTypeTok{sep =} \StringTok{","}\NormalTok{, }\DataTypeTok{header =} \OtherTok{TRUE}\NormalTok{)}
\NormalTok{dataset_}\DecValTok{1}\NormalTok{ <-}\StringTok{ }\NormalTok{dataset}
\KeywordTok{glimpse}\NormalTok{(dataset)}
\end{Highlighting}
\end{Shaded}

\begin{verbatim}
## Rows: 20,640
## Columns: 9
## $ median_house_value <dbl> 452600, 358500, 352100, 341300, 342200, 269700, ...
## $ median_income      <dbl> 8.3252, 8.3014, 7.2574, 5.6431, 3.8462, 4.0368, ...
## $ housing_median_age <dbl> 41, 21, 52, 52, 52, 52, 52, 52, 42, 52, 52, 52, ...
## $ total_rooms        <dbl> 880, 7099, 1467, 1274, 1627, 919, 2535, 3104, 25...
## $ total_bedrooms     <dbl> 129, 1106, 190, 235, 280, 213, 489, 687, 665, 70...
## $ population         <dbl> 322, 2401, 496, 558, 565, 413, 1094, 1157, 1206,...
## $ households         <dbl> 126, 1138, 177, 219, 259, 193, 514, 647, 595, 71...
## $ latitude           <dbl> 37.88, 37.86, 37.85, 37.85, 37.85, 37.85, 37.84,...
## $ longitude          <dbl> -122.23, -122.22, -122.24, -122.25, -122.25, -12...
\end{verbatim}

Hay nueve variables:

\begin{itemize}
\item
  \texttt{median\_house\_value}: Precio medio de la vivienda
\item
  \texttt{median\_income}: Ingreso medio de una familia
\item
  \texttt{housing\_median\_age}: Edad media de las viviendas
\item
  \texttt{total\_rooms}: Número total de cuartos
\item
  \texttt{total\_bedrooms}: Número total de habitaciones
\item
  \texttt{population}: Población del bloque
\item
  \texttt{households}: Familia que viven en el bloque
\item
  \texttt{latitude}: Latitud de la distancia entre los centroides de
  cada grupo de bloque
\item
  \texttt{longitude}: Longitud de la distancia entre los centroides de
  cada grupo de bloque
\end{itemize}

\begin{Shaded}
\begin{Highlighting}[]
\KeywordTok{summary}\NormalTok{(dataset)}
\end{Highlighting}
\end{Shaded}

\begin{verbatim}
##  median_house_value median_income     housing_median_age  total_rooms   
##  Min.   : 14999     Min.   : 0.4999   Min.   : 1.00      Min.   :    2  
##  1st Qu.:119600     1st Qu.: 2.5634   1st Qu.:18.00      1st Qu.: 1448  
##  Median :179700     Median : 3.5348   Median :29.00      Median : 2127  
##  Mean   :206856     Mean   : 3.8707   Mean   :28.64      Mean   : 2636  
##  3rd Qu.:264725     3rd Qu.: 4.7432   3rd Qu.:37.00      3rd Qu.: 3148  
##  Max.   :500001     Max.   :15.0001   Max.   :52.00      Max.   :39320  
##  total_bedrooms     population      households        latitude    
##  Min.   :   1.0   Min.   :    3   Min.   :   1.0   Min.   :32.54  
##  1st Qu.: 295.0   1st Qu.:  787   1st Qu.: 280.0   1st Qu.:33.93  
##  Median : 435.0   Median : 1166   Median : 409.0   Median :34.26  
##  Mean   : 537.9   Mean   : 1425   Mean   : 499.5   Mean   :35.63  
##  3rd Qu.: 647.0   3rd Qu.: 1725   3rd Qu.: 605.0   3rd Qu.:37.71  
##  Max.   :6445.0   Max.   :35682   Max.   :6082.0   Max.   :41.95  
##    longitude     
##  Min.   :-124.3  
##  1st Qu.:-121.8  
##  Median :-118.5  
##  Mean   :-119.6  
##  3rd Qu.:-118.0  
##  Max.   :-114.3
\end{verbatim}

\begin{Shaded}
\begin{Highlighting}[]
\KeywordTok{anyNA}\NormalTok{(dataset)}
\end{Highlighting}
\end{Shaded}

\begin{verbatim}
## [1] FALSE
\end{verbatim}

Afortunadamente no hay valores NA en nuestro conjunto de datos.

\begin{Shaded}
\begin{Highlighting}[]
\NormalTok{correlacion <-}\StringTok{ }\KeywordTok{cor}\NormalTok{(dataset, }
                   \DataTypeTok{method =} \StringTok{"spearman"}\NormalTok{)}

\KeywordTok{ggcorrplot}\NormalTok{(correlacion, }\DataTypeTok{lab =} \OtherTok{TRUE}\NormalTok{, }\DataTypeTok{lab_size =} \FloatTok{1.7}\NormalTok{, }\DataTypeTok{legend.title =} \StringTok{"Correlación", }
\StringTok{           lab_col = "}\NormalTok{blue4}\StringTok{", colors = c("}\NormalTok{yellow4}\StringTok{", "}\NormalTok{white}\StringTok{", "}\NormalTok{green4}\StringTok{")) +}
\StringTok{  theme_bw() +}
\StringTok{  theme(axis.text.x = element_text(angle = 45, vjust = 1, hjust=1)) +}
\StringTok{  labs(x = "", y = "", title = "}\NormalTok{Matriz de correlación")}
\end{Highlighting}
\end{Shaded}

\includegraphics{SVM_regresion_files/figure-latex/unnamed-chunk-4-1.pdf}

Hay muchas variables correlacionadas entre ellas pero no podemos
eliminar ninguna ya que en los modelos de regresión vamos a utilizarlas
todas (menos \texttt{latitude} y \texttt{longitude}).

Analizaremos los outliers presentes en cada variable para ver si los
dejamos o los eliminamos.

\hypertarget{variable-median_house_value}{%
\subsection{\texorpdfstring{Variable
\texttt{median\_house\_value}}{Variable median\_house\_value}}\label{variable-median_house_value}}

\begin{Shaded}
\begin{Highlighting}[]
\NormalTok{dataset }\OperatorTok
\StringTok{  }\KeywordTok{ggplot}\NormalTok{(}\KeywordTok{aes}\NormalTok{(}\DataTypeTok{x =} \DecValTok{0}\NormalTok{, }\DataTypeTok{y =}\NormalTok{ median_house_value)) }\OperatorTok{+}\StringTok{ }
\StringTok{  }\KeywordTok{geom_boxplot}\NormalTok{(}\DataTypeTok{width =} \FloatTok{0.7}\NormalTok{, }\DataTypeTok{outlier.colour =} \StringTok{"blue4"}\NormalTok{, }
               \DataTypeTok{outlier.size =} \DecValTok{2}\NormalTok{, }\DataTypeTok{outlier.shape =} \DecValTok{18}\NormalTok{) }\OperatorTok{+}
\StringTok{  }\KeywordTok{stat_summary}\NormalTok{(}\DataTypeTok{fun.y =}\NormalTok{ mean, }\DataTypeTok{geom =} \StringTok{"point"}\NormalTok{, }\DataTypeTok{shape =} \DecValTok{8}\NormalTok{, }\DataTypeTok{size =} \FloatTok{1.5}\NormalTok{, }\DataTypeTok{color =}\StringTok{"red4"}\NormalTok{) }\OperatorTok{+}
\StringTok{  }\KeywordTok{theme_linedraw}\NormalTok{() }\OperatorTok{+}
\StringTok{  }\KeywordTok{labs}\NormalTok{(}\DataTypeTok{x =} \StringTok{"x"}\NormalTok{, }\DataTypeTok{y =} \StringTok{"median_house_value"}\NormalTok{) }\OperatorTok{+}
\StringTok{  }\KeywordTok{theme}\NormalTok{(}\DataTypeTok{axis.text.x =} \KeywordTok{element_text}\NormalTok{(}\DataTypeTok{angle =} \DecValTok{45}\NormalTok{, }\DataTypeTok{vjust =} \DecValTok{1}\NormalTok{, }\DataTypeTok{hjust=}\DecValTok{1}\NormalTok{))}
\end{Highlighting}
\end{Shaded}

\includegraphics{SVM_regresion_files/figure-latex/unnamed-chunk-5-1.pdf}

\begin{Shaded}
\begin{Highlighting}[]
\KeywordTok{table}\NormalTok{(dataset}\OperatorTok{$}\NormalTok{median_house_value }\OperatorTok{>}\StringTok{ }\DecValTok{400000}\NormalTok{)}
\end{Highlighting}
\end{Shaded}

\begin{verbatim}
## 
## FALSE  TRUE 
## 18896  1744
\end{verbatim}

Vemos que los puntos outliers los tenemos por encima de 400000\$ y hay
1744 observaciones. No los quitaremos ya que hay muchas y se trata de la
variable objetivo.

Hay 1744 outliers en esta variable.

\hypertarget{variable-median_income}{%
\subsection{\texorpdfstring{Variable
\texttt{median\_income}}{Variable median\_income}}\label{variable-median_income}}

\begin{Shaded}
\begin{Highlighting}[]
\NormalTok{dataset }\OperatorTok
\StringTok{  }\KeywordTok{ggplot}\NormalTok{(}\KeywordTok{aes}\NormalTok{(}\DataTypeTok{x =} \DecValTok{0}\NormalTok{, }\DataTypeTok{y =}\NormalTok{ median_income)) }\OperatorTok{+}\StringTok{ }
\StringTok{  }\KeywordTok{geom_boxplot}\NormalTok{(}\DataTypeTok{width =} \FloatTok{0.7}\NormalTok{, }\DataTypeTok{outlier.colour =} \StringTok{"blue4"}\NormalTok{, }
               \DataTypeTok{outlier.size =} \DecValTok{2}\NormalTok{, }\DataTypeTok{outlier.shape =} \DecValTok{18}\NormalTok{) }\OperatorTok{+}
\StringTok{  }\KeywordTok{stat_summary}\NormalTok{(}\DataTypeTok{fun.y =}\NormalTok{ mean, }\DataTypeTok{geom =} \StringTok{"point"}\NormalTok{, }\DataTypeTok{shape =} \DecValTok{8}\NormalTok{, }\DataTypeTok{size =} \FloatTok{1.5}\NormalTok{, }\DataTypeTok{color =}\StringTok{"red4"}\NormalTok{) }\OperatorTok{+}
\StringTok{  }\KeywordTok{theme_linedraw}\NormalTok{() }\OperatorTok{+}
\StringTok{  }\KeywordTok{labs}\NormalTok{(}\DataTypeTok{x =} \StringTok{"x"}\NormalTok{, }\DataTypeTok{y =} \StringTok{"median_income"}\NormalTok{) }\OperatorTok{+}
\StringTok{  }\KeywordTok{theme}\NormalTok{(}\DataTypeTok{axis.text.x =} \KeywordTok{element_text}\NormalTok{(}\DataTypeTok{angle =} \DecValTok{45}\NormalTok{, }\DataTypeTok{vjust =} \DecValTok{1}\NormalTok{, }\DataTypeTok{hjust=}\DecValTok{1}\NormalTok{))}
\end{Highlighting}
\end{Shaded}

\includegraphics{SVM_regresion_files/figure-latex/unnamed-chunk-7-1.pdf}

\begin{Shaded}
\begin{Highlighting}[]
\KeywordTok{table}\NormalTok{(dataset}\OperatorTok{$}\NormalTok{median_income }\OperatorTok{>}\StringTok{ }\FloatTok{12.5}\NormalTok{)}
\end{Highlighting}
\end{Shaded}

\begin{verbatim}
## 
## FALSE  TRUE 
## 20546    94
\end{verbatim}

En este caso, vemos que los outliers son los bloques que tienen como
ingreso medio por familia por encima de 7.5. Sin embargo quitaremos los
que estén por encima de 12.5 ya que si no quitaríamos muchas
observaciones.

\begin{Shaded}
\begin{Highlighting}[]
\NormalTok{dataset_}\DecValTok{1}\NormalTok{ <-}\StringTok{ }\NormalTok{dataset_}\DecValTok{1} \OperatorTok
\StringTok{  }\KeywordTok{filter}\NormalTok{(dataset_}\DecValTok{1}\OperatorTok{$}\NormalTok{median_income }\OperatorTok{<=}\StringTok{ }\FloatTok{12.5}\NormalTok{)}
\end{Highlighting}
\end{Shaded}

\hypertarget{variable-housing_median_age}{%
\subsection{\texorpdfstring{Variable
\texttt{housing\_median\_age}}{Variable housing\_median\_age}}\label{variable-housing_median_age}}

\begin{Shaded}
\begin{Highlighting}[]
\NormalTok{dataset }\OperatorTok
\StringTok{  }\KeywordTok{ggplot}\NormalTok{(}\KeywordTok{aes}\NormalTok{(}\DataTypeTok{x =} \DecValTok{0}\NormalTok{, }\DataTypeTok{y =}\NormalTok{ housing_median_age)) }\OperatorTok{+}\StringTok{ }
\StringTok{  }\KeywordTok{geom_boxplot}\NormalTok{(}\DataTypeTok{width =} \FloatTok{0.7}\NormalTok{, }\DataTypeTok{outlier.colour =} \StringTok{"blue4"}\NormalTok{, }
               \DataTypeTok{outlier.size =} \DecValTok{2}\NormalTok{, }\DataTypeTok{outlier.shape =} \DecValTok{18}\NormalTok{) }\OperatorTok{+}
\StringTok{  }\KeywordTok{stat_summary}\NormalTok{(}\DataTypeTok{fun.y =}\NormalTok{ mean, }\DataTypeTok{geom =} \StringTok{"point"}\NormalTok{, }\DataTypeTok{shape =} \DecValTok{8}\NormalTok{, }\DataTypeTok{size =} \FloatTok{1.5}\NormalTok{, }\DataTypeTok{color =}\StringTok{"red4"}\NormalTok{) }\OperatorTok{+}
\StringTok{  }\KeywordTok{theme_linedraw}\NormalTok{() }\OperatorTok{+}
\StringTok{  }\KeywordTok{labs}\NormalTok{(}\DataTypeTok{x =} \StringTok{"x"}\NormalTok{, }\DataTypeTok{y =} \StringTok{"housing_median_age"}\NormalTok{) }\OperatorTok{+}
\StringTok{  }\KeywordTok{theme}\NormalTok{(}\DataTypeTok{axis.text.x =} \KeywordTok{element_text}\NormalTok{(}\DataTypeTok{angle =} \DecValTok{45}\NormalTok{, }\DataTypeTok{vjust =} \DecValTok{1}\NormalTok{, }\DataTypeTok{hjust=}\DecValTok{1}\NormalTok{))}
\end{Highlighting}
\end{Shaded}

\includegraphics{SVM_regresion_files/figure-latex/unnamed-chunk-10-1.pdf}

No hay ningún outlier para esta variable, entonces no quitaremos ningún
dato.

\hypertarget{variable-total_rooms}{%
\subsection{\texorpdfstring{Variable
\texttt{total\_rooms}}{Variable total\_rooms}}\label{variable-total_rooms}}

\begin{Shaded}
\begin{Highlighting}[]
\NormalTok{dataset }\OperatorTok
\StringTok{  }\KeywordTok{ggplot}\NormalTok{(}\KeywordTok{aes}\NormalTok{(}\DataTypeTok{x =} \DecValTok{0}\NormalTok{, }\DataTypeTok{y =}\NormalTok{ total_rooms)) }\OperatorTok{+}\StringTok{ }
\StringTok{  }\KeywordTok{geom_boxplot}\NormalTok{(}\DataTypeTok{width =} \FloatTok{0.7}\NormalTok{, }\DataTypeTok{outlier.colour =} \StringTok{"blue4"}\NormalTok{, }
               \DataTypeTok{outlier.size =} \DecValTok{2}\NormalTok{, }\DataTypeTok{outlier.shape =} \DecValTok{18}\NormalTok{) }\OperatorTok{+}
\StringTok{  }\KeywordTok{stat_summary}\NormalTok{(}\DataTypeTok{fun.y =}\NormalTok{ mean, }\DataTypeTok{geom =} \StringTok{"point"}\NormalTok{, }\DataTypeTok{shape =} \DecValTok{8}\NormalTok{, }\DataTypeTok{size =} \FloatTok{1.5}\NormalTok{, }\DataTypeTok{color =}\StringTok{"red4"}\NormalTok{) }\OperatorTok{+}
\StringTok{  }\KeywordTok{theme_linedraw}\NormalTok{() }\OperatorTok{+}
\StringTok{  }\KeywordTok{labs}\NormalTok{(}\DataTypeTok{x =} \StringTok{"x"}\NormalTok{, }\DataTypeTok{y =} \StringTok{"total_rooms"}\NormalTok{) }\OperatorTok{+}
\StringTok{  }\KeywordTok{theme}\NormalTok{(}\DataTypeTok{axis.text.x =} \KeywordTok{element_text}\NormalTok{(}\DataTypeTok{angle =} \DecValTok{45}\NormalTok{, }\DataTypeTok{vjust =} \DecValTok{1}\NormalTok{, }\DataTypeTok{hjust=}\DecValTok{1}\NormalTok{))}
\end{Highlighting}
\end{Shaded}

\includegraphics{SVM_regresion_files/figure-latex/unnamed-chunk-11-1.pdf}

\begin{Shaded}
\begin{Highlighting}[]
\KeywordTok{table}\NormalTok{(dataset}\OperatorTok{$}\NormalTok{total_rooms }\OperatorTok{>}\StringTok{ }\DecValTok{30000}\NormalTok{)}
\end{Highlighting}
\end{Shaded}

\begin{verbatim}
## 
## FALSE  TRUE 
## 20633     7
\end{verbatim}

El número total de cuartos varía mucho dependiendo de la ubicación y el
tipo de cada bloque, con lo cual hay una gran cantidad de outliers.
Consideraremos los bloques que tienen más de 30000 cuartos como outliers
y los eliminaremos.

\begin{Shaded}
\begin{Highlighting}[]
\NormalTok{dataset_}\DecValTok{1}\NormalTok{ <-}\StringTok{ }\NormalTok{dataset_}\DecValTok{1} \OperatorTok
\StringTok{  }\KeywordTok{filter}\NormalTok{(dataset_}\DecValTok{1}\OperatorTok{$}\NormalTok{total_rooms }\OperatorTok{<=}\StringTok{ }\DecValTok{30000}\NormalTok{)}
\end{Highlighting}
\end{Shaded}

\hypertarget{variable-total_bedrooms}{%
\subsection{\texorpdfstring{Variable
\texttt{total\_bedrooms}}{Variable total\_bedrooms}}\label{variable-total_bedrooms}}

\begin{Shaded}
\begin{Highlighting}[]
\NormalTok{dataset }\OperatorTok
\StringTok{  }\KeywordTok{ggplot}\NormalTok{(}\KeywordTok{aes}\NormalTok{(}\DataTypeTok{x =} \DecValTok{0}\NormalTok{, }\DataTypeTok{y =}\NormalTok{ total_bedrooms)) }\OperatorTok{+}\StringTok{ }
\StringTok{  }\KeywordTok{geom_boxplot}\NormalTok{(}\DataTypeTok{width =} \FloatTok{0.7}\NormalTok{, }\DataTypeTok{outlier.colour =} \StringTok{"blue4"}\NormalTok{, }
               \DataTypeTok{outlier.size =} \DecValTok{2}\NormalTok{, }\DataTypeTok{outlier.shape =} \DecValTok{18}\NormalTok{) }\OperatorTok{+}
\StringTok{  }\KeywordTok{stat_summary}\NormalTok{(}\DataTypeTok{fun.y =}\NormalTok{ mean, }\DataTypeTok{geom =} \StringTok{"point"}\NormalTok{, }\DataTypeTok{shape =} \DecValTok{8}\NormalTok{, }\DataTypeTok{size =} \FloatTok{1.5}\NormalTok{, }\DataTypeTok{color =}\StringTok{"red4"}\NormalTok{) }\OperatorTok{+}
\StringTok{  }\KeywordTok{theme_linedraw}\NormalTok{() }\OperatorTok{+}
\StringTok{  }\KeywordTok{labs}\NormalTok{(}\DataTypeTok{x =} \StringTok{"x"}\NormalTok{, }\DataTypeTok{y =} \StringTok{"total_bedrooms"}\NormalTok{) }\OperatorTok{+}
\StringTok{  }\KeywordTok{theme}\NormalTok{(}\DataTypeTok{axis.text.x =} \KeywordTok{element_text}\NormalTok{(}\DataTypeTok{angle =} \DecValTok{45}\NormalTok{, }\DataTypeTok{vjust =} \DecValTok{1}\NormalTok{, }\DataTypeTok{hjust=}\DecValTok{1}\NormalTok{))}
\end{Highlighting}
\end{Shaded}

\includegraphics{SVM_regresion_files/figure-latex/unnamed-chunk-14-1.pdf}

\begin{Shaded}
\begin{Highlighting}[]
\KeywordTok{table}\NormalTok{(dataset}\OperatorTok{$}\NormalTok{total_bedrooms }\OperatorTok{>}\StringTok{ }\DecValTok{4000}\NormalTok{)}
\end{Highlighting}
\end{Shaded}

\begin{verbatim}
## 
## FALSE  TRUE 
## 20614    26
\end{verbatim}

Igual que el caso anterior, quitaremos las 26 observaciones que tienen
más de 4000 habitaciones ya que sería observaciones muy atípicas.

\begin{Shaded}
\begin{Highlighting}[]
\NormalTok{dataset_}\DecValTok{1}\NormalTok{ <-}\StringTok{ }\NormalTok{dataset_}\DecValTok{1} \OperatorTok
\StringTok{  }\KeywordTok{filter}\NormalTok{(dataset_}\DecValTok{1}\OperatorTok{$}\NormalTok{total_bedrooms }\OperatorTok{<=}\StringTok{ }\DecValTok{4000}\NormalTok{)}
\end{Highlighting}
\end{Shaded}

\hypertarget{variable-population}{%
\subsection{\texorpdfstring{Variable
\texttt{population}}{Variable population}}\label{variable-population}}

\begin{Shaded}
\begin{Highlighting}[]
\NormalTok{dataset }\OperatorTok
\StringTok{  }\KeywordTok{ggplot}\NormalTok{(}\KeywordTok{aes}\NormalTok{(}\DataTypeTok{x =} \DecValTok{0}\NormalTok{, }\DataTypeTok{y =}\NormalTok{ population)) }\OperatorTok{+}\StringTok{ }
\StringTok{  }\KeywordTok{geom_boxplot}\NormalTok{(}\DataTypeTok{width =} \FloatTok{0.7}\NormalTok{, }\DataTypeTok{outlier.colour =} \StringTok{"blue4"}\NormalTok{, }
               \DataTypeTok{outlier.size =} \DecValTok{2}\NormalTok{, }\DataTypeTok{outlier.shape =} \DecValTok{18}\NormalTok{) }\OperatorTok{+}
\StringTok{  }\KeywordTok{stat_summary}\NormalTok{(}\DataTypeTok{fun.y =}\NormalTok{ mean, }\DataTypeTok{geom =} \StringTok{"point"}\NormalTok{, }\DataTypeTok{shape =} \DecValTok{8}\NormalTok{, }\DataTypeTok{size =} \FloatTok{1.5}\NormalTok{, }\DataTypeTok{color =}\StringTok{"red4"}\NormalTok{) }\OperatorTok{+}
\StringTok{  }\KeywordTok{theme_linedraw}\NormalTok{() }\OperatorTok{+}
\StringTok{  }\KeywordTok{labs}\NormalTok{(}\DataTypeTok{x =} \StringTok{"x"}\NormalTok{, }\DataTypeTok{y =} \StringTok{"population"}\NormalTok{) }\OperatorTok{+}
\StringTok{  }\KeywordTok{theme}\NormalTok{(}\DataTypeTok{axis.text.x =} \KeywordTok{element_text}\NormalTok{(}\DataTypeTok{angle =} \DecValTok{45}\NormalTok{, }\DataTypeTok{vjust =} \DecValTok{1}\NormalTok{, }\DataTypeTok{hjust=}\DecValTok{1}\NormalTok{))}
\end{Highlighting}
\end{Shaded}

\includegraphics{SVM_regresion_files/figure-latex/unnamed-chunk-17-1.pdf}

\begin{Shaded}
\begin{Highlighting}[]
\KeywordTok{table}\NormalTok{(dataset}\OperatorTok{$}\NormalTok{population }\OperatorTok{>}\StringTok{ }\DecValTok{20000}\NormalTok{)}
\end{Highlighting}
\end{Shaded}

\begin{verbatim}
## 
## FALSE  TRUE 
## 20638     2
\end{verbatim}

La población es importante. Consideraremos los bloques que tienen más de
20000 habitantes como outliers y los eliminaremos.

\begin{Shaded}
\begin{Highlighting}[]
\NormalTok{dataset_}\DecValTok{1}\NormalTok{ <-}\StringTok{ }\NormalTok{dataset_}\DecValTok{1} \OperatorTok
\StringTok{  }\KeywordTok{filter}\NormalTok{(dataset_}\DecValTok{1}\OperatorTok{$}\NormalTok{population }\OperatorTok{<=}\StringTok{ }\DecValTok{20000}\NormalTok{)}
\end{Highlighting}
\end{Shaded}

\hypertarget{variable-households}{%
\subsection{\texorpdfstring{Variable
\texttt{households}}{Variable households}}\label{variable-households}}

\begin{Shaded}
\begin{Highlighting}[]
\NormalTok{dataset }\OperatorTok
\StringTok{  }\KeywordTok{ggplot}\NormalTok{(}\KeywordTok{aes}\NormalTok{(}\DataTypeTok{x =} \DecValTok{0}\NormalTok{, }\DataTypeTok{y =}\NormalTok{ households)) }\OperatorTok{+}\StringTok{ }
\StringTok{  }\KeywordTok{geom_boxplot}\NormalTok{(}\DataTypeTok{width =} \FloatTok{0.7}\NormalTok{, }\DataTypeTok{outlier.colour =} \StringTok{"blue4"}\NormalTok{, }
               \DataTypeTok{outlier.size =} \DecValTok{2}\NormalTok{, }\DataTypeTok{outlier.shape =} \DecValTok{18}\NormalTok{) }\OperatorTok{+}
\StringTok{  }\KeywordTok{stat_summary}\NormalTok{(}\DataTypeTok{fun.y =}\NormalTok{ mean, }\DataTypeTok{geom =} \StringTok{"point"}\NormalTok{, }\DataTypeTok{shape =} \DecValTok{8}\NormalTok{, }\DataTypeTok{size =} \FloatTok{1.5}\NormalTok{, }\DataTypeTok{color =}\StringTok{"red4"}\NormalTok{) }\OperatorTok{+}
\StringTok{  }\KeywordTok{theme_linedraw}\NormalTok{() }\OperatorTok{+}
\StringTok{  }\KeywordTok{labs}\NormalTok{(}\DataTypeTok{x =} \StringTok{"x"}\NormalTok{, }\DataTypeTok{y =} \StringTok{"households"}\NormalTok{) }\OperatorTok{+}
\StringTok{  }\KeywordTok{theme}\NormalTok{(}\DataTypeTok{axis.text.x =} \KeywordTok{element_text}\NormalTok{(}\DataTypeTok{angle =} \DecValTok{45}\NormalTok{, }\DataTypeTok{vjust =} \DecValTok{1}\NormalTok{, }\DataTypeTok{hjust=}\DecValTok{1}\NormalTok{))}
\end{Highlighting}
\end{Shaded}

\includegraphics{SVM_regresion_files/figure-latex/unnamed-chunk-20-1.pdf}

\begin{Shaded}
\begin{Highlighting}[]
\KeywordTok{table}\NormalTok{(dataset}\OperatorTok{$}\NormalTok{households }\OperatorTok{>}\StringTok{ }\DecValTok{4000}\NormalTok{)}
\end{Highlighting}
\end{Shaded}

\begin{verbatim}
## 
## FALSE  TRUE 
## 20625    15
\end{verbatim}

El número de familia que viven en un bloque que son considerados
outliers es por encima 1000. Pero consideraremos eliminar los bloques
que tienen más de 4000 familias viviendo en ese bloque.

\begin{Shaded}
\begin{Highlighting}[]
\NormalTok{dataset_}\DecValTok{1}\NormalTok{ <-}\StringTok{ }\NormalTok{dataset_}\DecValTok{1} \OperatorTok
\StringTok{  }\KeywordTok{filter}\NormalTok{(dataset_}\DecValTok{1}\OperatorTok{$}\NormalTok{households }\OperatorTok{<=}\StringTok{ }\DecValTok{4000}\NormalTok{)}
\end{Highlighting}
\end{Shaded}

\hypertarget{svr}{%
\section{SVR}\label{svr}}

El modelo que vamos a construir es el siguiente:

\[\begin{equation*}
        \begin{split}
            \mbox{ln(median house value)} &= a_1 + a_2\mbox{median income} + a_3\mbox{median income}^2 + a_4\mbox{median income}^3 + a_5\mbox{ln(housing median_age)} \\
            & + a_6ln(\frac{\mbox{total rooms}}{population}) + a_7ln(\frac{\mbox{total bedrooms}}{population}) a_8ln(\frac{population}{households}) + a_9ln(households)\\
        \end{split}
\end{equation*}\]

Entonces vamos a construir un data frame que contenga estas variables y
eliminamos las originales.

\begin{Shaded}
\begin{Highlighting}[]
\NormalTok{dataset_}\DecValTok{1}\NormalTok{ <-}\StringTok{ }\NormalTok{dataset_}\DecValTok{1} \OperatorTok
\StringTok{  }\KeywordTok{mutate}\NormalTok{(}\DataTypeTok{ln_median_house_value =} \KeywordTok{log}\NormalTok{(median_house_value), }\DataTypeTok{median_income_2 =}\NormalTok{ median_income}\OperatorTok{^}\DecValTok{2}\NormalTok{, }
         \DataTypeTok{median_income_3 =}\NormalTok{ median_income}\OperatorTok{^}\DecValTok{3}\NormalTok{, }
         \DataTypeTok{ln_housing_median_age =} \KeywordTok{log}\NormalTok{(housing_median_age), }\DataTypeTok{ln_total_rooms_population =}
           \KeywordTok{log}\NormalTok{(total_rooms}\OperatorTok{/}\NormalTok{population), }\DataTypeTok{ln_total_bedrooms_population =}
           \KeywordTok{log}\NormalTok{(total_bedrooms}\OperatorTok{/}\NormalTok{population), }\DataTypeTok{ln_population_households =}
           \KeywordTok{log}\NormalTok{(population}\OperatorTok{/}\NormalTok{households), }\DataTypeTok{ln_households =} \KeywordTok{log}\NormalTok{(households))}

\NormalTok{dataset_}\DecValTok{1}\NormalTok{ <-}\StringTok{ }\NormalTok{dataset_}\DecValTok{1} \OperatorTok
\StringTok{  }\NormalTok{dplyr}\OperatorTok{::}\KeywordTok{select}\NormalTok{(}\KeywordTok{c}\NormalTok{(}\StringTok{"median_income"}\NormalTok{, }\StringTok{"ln_median_house_value"}\NormalTok{, }\StringTok{"median_income_2"}\NormalTok{, }\StringTok{"median_income_3"}\NormalTok{,}
           \StringTok{"ln_housing_median_age"}\NormalTok{, }\StringTok{"ln_total_rooms_population"}\NormalTok{, }\StringTok{"ln_total_bedrooms_population"}\NormalTok{,}
           \StringTok{"ln_population_households"}\NormalTok{, }\StringTok{"ln_households"}\NormalTok{))}
\end{Highlighting}
\end{Shaded}

Una vez que tenemos nuestro data set con las variables involucradas en
la creación de los modelos que haremos, vamos a normalizar los datos
menos la variable objetivo.

\begin{Shaded}
\begin{Highlighting}[]
\NormalTok{normalize <-}\StringTok{ }\ControlFlowTok{function}\NormalTok{(x)}
\NormalTok{  \{}
    \KeywordTok{return}\NormalTok{((x}\OperatorTok{-}\StringTok{ }\KeywordTok{min}\NormalTok{(x)) }\OperatorTok{/}\NormalTok{(}\KeywordTok{max}\NormalTok{(x)}\OperatorTok{-}\KeywordTok{min}\NormalTok{(x)))}
\NormalTok{  \}}

\NormalTok{dataset_norm <-}\StringTok{ }\KeywordTok{as.data.frame}\NormalTok{(}\KeywordTok{lapply}\NormalTok{(dataset_}\DecValTok{1}\NormalTok{[,}\OperatorTok{-}\DecValTok{2}\NormalTok{], normalize))}
\NormalTok{dataset_norm <-}\StringTok{ }\NormalTok{dataset_norm }\OperatorTok
\StringTok{  }\KeywordTok{mutate}\NormalTok{(}\DataTypeTok{ln_median_house_value =}\NormalTok{ dataset_}\DecValTok{1}\OperatorTok{$}\NormalTok{ln_median_house_value)}
\end{Highlighting}
\end{Shaded}

Separamos nuestro conjunto de datos en dos: conjunto de entrenamiento y
conjunto de validación con un ratio de 70:30.

\begin{Shaded}
\begin{Highlighting}[]
\KeywordTok{set.seed}\NormalTok{(}\DecValTok{284}\NormalTok{)}
\NormalTok{split <-}\StringTok{ }\KeywordTok{sample.split}\NormalTok{(dataset_norm}\OperatorTok{$}\NormalTok{ln_median_house_value, }\DataTypeTok{SplitRatio =} \FloatTok{0.7}\NormalTok{)}
\NormalTok{dt.train <-}\StringTok{ }\KeywordTok{subset}\NormalTok{(dataset_norm, split }\OperatorTok{==}\StringTok{ }\OtherTok{TRUE}\NormalTok{)}
\NormalTok{dt.test <-}\StringTok{ }\KeywordTok{subset}\NormalTok{(dataset_norm, split }\OperatorTok{==}\StringTok{ }\OtherTok{FALSE}\NormalTok{)}
\end{Highlighting}
\end{Shaded}

Los errores que conseguimos en cada modelo los guardamos en estos dos
vectores, dependiendo de si se trata de Epsilon-SVR o Nu-SVR.

\begin{Shaded}
\begin{Highlighting}[]
\NormalTok{eps_reg <-}\StringTok{ }\KeywordTok{vector}\NormalTok{()}
\NormalTok{nu_reg <-}\StringTok{ }\KeywordTok{vector}\NormalTok{()}
\end{Highlighting}
\end{Shaded}

Vamos a realizar un total de 32 modelos de SVR. Para cada tipo de SVR
(Epsilon-SVR y Nu-SVR) realizaremos modelos para los tres tipos de
kernel: lineal, radial y polinomial (grado 2 y 3); y para cada uno de
estos tipos de kernel haremos 4 modelos con estos valores del parámetro
coste: \(\{0.001,0.01,0.1,1\}\). Utilzaremos la raíz del error
cuadrático medio como la medida para valorar los modelos.

\hypertarget{epsilon-svr}{%
\subsection{Epsilon-SVR}\label{epsilon-svr}}

\hypertarget{kernel-lineal}{%
\subsubsection{Kernel lineal}\label{kernel-lineal}}

\begin{Shaded}
\begin{Highlighting}[]
\NormalTok{svr_}\DecValTok{1}\NormalTok{ <-}\StringTok{ }\KeywordTok{svm}\NormalTok{(ln_median_house_value }\OperatorTok{~}\NormalTok{. , }
                 \DataTypeTok{data =}\NormalTok{ dt.train, }
                 \DataTypeTok{type =} \StringTok{"eps-regression"}\NormalTok{, }
                 \DataTypeTok{kernel =} \StringTok{"linear"}\NormalTok{, }\DataTypeTok{cost =} \FloatTok{0.001}\NormalTok{)}
\NormalTok{svr_pred_}\DecValTok{1}\NormalTok{ <-}\StringTok{ }\KeywordTok{predict}\NormalTok{(svr_}\DecValTok{1}\NormalTok{, }\DataTypeTok{newdata =}\NormalTok{ dt.test)}
\NormalTok{rmse_svr_}\DecValTok{1}\NormalTok{ <-}\StringTok{ }\KeywordTok{sqrt}\NormalTok{(}\KeywordTok{MSE}\NormalTok{(}\DataTypeTok{y_pred =}\NormalTok{ svr_pred_}\DecValTok{1}\NormalTok{, }\DataTypeTok{y_true =}\NormalTok{ dt.test}\OperatorTok{$}\NormalTok{ln_median_house_value))}
\NormalTok{eps_reg[}\DecValTok{1}\NormalTok{] <-}\StringTok{ }\NormalTok{rmse_svr_}\DecValTok{1}
\NormalTok{rmse_svr_}\DecValTok{1}
\end{Highlighting}
\end{Shaded}

\begin{verbatim}
## [1] 0.364327
\end{verbatim}

\begin{Shaded}
\begin{Highlighting}[]
\NormalTok{svr_}\DecValTok{2}\NormalTok{ <-}\StringTok{ }\KeywordTok{svm}\NormalTok{(ln_median_house_value }\OperatorTok{~}\NormalTok{. , }
                 \DataTypeTok{data =}\NormalTok{ dt.train, }
                 \DataTypeTok{type =} \StringTok{"eps-regression"}\NormalTok{, }
                 \DataTypeTok{kernel =} \StringTok{"linear"}\NormalTok{, }\DataTypeTok{cost =} \FloatTok{0.01}\NormalTok{)}
\NormalTok{svr_pred_}\DecValTok{2}\NormalTok{ <-}\StringTok{ }\KeywordTok{predict}\NormalTok{(svr_}\DecValTok{2}\NormalTok{, }\DataTypeTok{newdata =}\NormalTok{ dt.test)}
\NormalTok{rmse_svr_}\DecValTok{2}\NormalTok{ <-}\StringTok{ }\KeywordTok{sqrt}\NormalTok{(}\KeywordTok{MSE}\NormalTok{(}\DataTypeTok{y_pred =}\NormalTok{ svr_pred_}\DecValTok{2}\NormalTok{, }\DataTypeTok{y_true =}\NormalTok{ dt.test}\OperatorTok{$}\NormalTok{ln_median_house_value))}
\NormalTok{eps_reg[}\DecValTok{2}\NormalTok{] <-}\StringTok{ }\NormalTok{rmse_svr_}\DecValTok{2}
\NormalTok{rmse_svr_}\DecValTok{2}
\end{Highlighting}
\end{Shaded}

\begin{verbatim}
## [1] 0.3553593
\end{verbatim}

\begin{Shaded}
\begin{Highlighting}[]
\NormalTok{svr_}\DecValTok{3}\NormalTok{ <-}\StringTok{ }\KeywordTok{svm}\NormalTok{(ln_median_house_value }\OperatorTok{~}\NormalTok{. , }
                 \DataTypeTok{data =}\NormalTok{ dt.train, }
                 \DataTypeTok{type =} \StringTok{"eps-regression"}\NormalTok{, }
                 \DataTypeTok{kernel =} \StringTok{"linear"}\NormalTok{, }\DataTypeTok{cost =} \FloatTok{0.1}\NormalTok{)}
\NormalTok{svr_pred_}\DecValTok{3}\NormalTok{ <-}\StringTok{ }\KeywordTok{predict}\NormalTok{(svr_}\DecValTok{3}\NormalTok{, }\DataTypeTok{newdata =}\NormalTok{ dt.test)}
\NormalTok{rmse_svr_}\DecValTok{3}\NormalTok{ <-}\StringTok{ }\KeywordTok{sqrt}\NormalTok{(}\KeywordTok{MSE}\NormalTok{(}\DataTypeTok{y_pred =}\NormalTok{ svr_pred_}\DecValTok{3}\NormalTok{, }\DataTypeTok{y_true =}\NormalTok{ dt.test}\OperatorTok{$}\NormalTok{ln_median_house_value))}
\NormalTok{eps_reg[}\DecValTok{3}\NormalTok{] <-}\StringTok{ }\NormalTok{rmse_svr_}\DecValTok{3}
\NormalTok{rmse_svr_}\DecValTok{3}
\end{Highlighting}
\end{Shaded}

\begin{verbatim}
## [1] 0.3562273
\end{verbatim}

\begin{Shaded}
\begin{Highlighting}[]
\NormalTok{svr_}\DecValTok{4}\NormalTok{ <-}\StringTok{ }\KeywordTok{svm}\NormalTok{(ln_median_house_value }\OperatorTok{~}\NormalTok{. , }
                 \DataTypeTok{data =}\NormalTok{ dt.train, }
                 \DataTypeTok{type =} \StringTok{"eps-regression"}\NormalTok{, }
                 \DataTypeTok{kernel =} \StringTok{"linear"}\NormalTok{, }\DataTypeTok{cost =} \DecValTok{1}\NormalTok{)}
\NormalTok{svr_pred_}\DecValTok{4}\NormalTok{ <-}\StringTok{ }\KeywordTok{predict}\NormalTok{(svr_}\DecValTok{4}\NormalTok{, }\DataTypeTok{newdata =}\NormalTok{ dt.test)}
\NormalTok{rmse_svr_}\DecValTok{4}\NormalTok{ <-}\StringTok{ }\KeywordTok{sqrt}\NormalTok{(}\KeywordTok{MSE}\NormalTok{(}\DataTypeTok{y_pred =}\NormalTok{ svr_pred_}\DecValTok{4}\NormalTok{, }\DataTypeTok{y_true =}\NormalTok{ dt.test}\OperatorTok{$}\NormalTok{ln_median_house_value))}
\NormalTok{eps_reg[}\DecValTok{4}\NormalTok{] <-}\StringTok{ }\NormalTok{rmse_svr_}\DecValTok{4}
\NormalTok{rmse_svr_}\DecValTok{4}
\end{Highlighting}
\end{Shaded}

\begin{verbatim}
## [1] 0.3564667
\end{verbatim}

\hypertarget{kernel-radial}{%
\subsubsection{Kernel radial}\label{kernel-radial}}

\begin{Shaded}
\begin{Highlighting}[]
\NormalTok{svr_}\DecValTok{5}\NormalTok{ <-}\StringTok{ }\KeywordTok{svm}\NormalTok{(ln_median_house_value }\OperatorTok{~}\NormalTok{. , }
                 \DataTypeTok{data =}\NormalTok{ dt.train, }
                 \DataTypeTok{type =} \StringTok{"eps-regression"}\NormalTok{, }
                 \DataTypeTok{kernel =} \StringTok{"radial"}\NormalTok{, }\DataTypeTok{cost =} \FloatTok{0.001}\NormalTok{, }\DataTypeTok{gamma =} \FloatTok{0.1}\NormalTok{)}
\NormalTok{svr_pred_}\DecValTok{5}\NormalTok{ <-}\StringTok{ }\KeywordTok{predict}\NormalTok{(svr_}\DecValTok{5}\NormalTok{, }\DataTypeTok{newdata =}\NormalTok{ dt.test)}
\NormalTok{rmse_svr_}\DecValTok{5}\NormalTok{ <-}\StringTok{ }\KeywordTok{sqrt}\NormalTok{(}\KeywordTok{MSE}\NormalTok{(}\DataTypeTok{y_pred =}\NormalTok{ svr_pred_}\DecValTok{5}\NormalTok{, }\DataTypeTok{y_true =}\NormalTok{ dt.test}\OperatorTok{$}\NormalTok{ln_median_house_value))}
\NormalTok{eps_reg[}\DecValTok{5}\NormalTok{] <-}\StringTok{ }\NormalTok{rmse_svr_}\DecValTok{5}
\NormalTok{rmse_svr_}\DecValTok{5}
\end{Highlighting}
\end{Shaded}

\begin{verbatim}
## [1] 0.430074
\end{verbatim}

\begin{Shaded}
\begin{Highlighting}[]
\NormalTok{svr_}\DecValTok{6}\NormalTok{ <-}\StringTok{ }\KeywordTok{svm}\NormalTok{(ln_median_house_value }\OperatorTok{~}\NormalTok{. , }
                 \DataTypeTok{data =}\NormalTok{ dt.train, }
                 \DataTypeTok{type =} \StringTok{"eps-regression"}\NormalTok{, }
                 \DataTypeTok{kernel =} \StringTok{"radial"}\NormalTok{, }\DataTypeTok{cost =} \FloatTok{0.01}\NormalTok{, }\DataTypeTok{gamma =} \FloatTok{0.1}\NormalTok{)}
\NormalTok{svr_pred_}\DecValTok{6}\NormalTok{ <-}\StringTok{ }\KeywordTok{predict}\NormalTok{(svr_}\DecValTok{5}\NormalTok{, }\DataTypeTok{newdata =}\NormalTok{ dt.test)}
\NormalTok{rmse_svr_}\DecValTok{6}\NormalTok{ <-}\StringTok{ }\KeywordTok{sqrt}\NormalTok{(}\KeywordTok{MSE}\NormalTok{(}\DataTypeTok{y_pred =}\NormalTok{ svr_pred_}\DecValTok{6}\NormalTok{, }\DataTypeTok{y_true =}\NormalTok{ dt.test}\OperatorTok{$}\NormalTok{ln_median_house_value))}
\NormalTok{eps_reg[}\DecValTok{6}\NormalTok{] <-}\StringTok{ }\NormalTok{rmse_svr_}\DecValTok{6}
\NormalTok{rmse_svr_}\DecValTok{6}
\end{Highlighting}
\end{Shaded}

\begin{verbatim}
## [1] 0.430074
\end{verbatim}

\begin{Shaded}
\begin{Highlighting}[]
\NormalTok{svr_}\DecValTok{7}\NormalTok{ <-}\StringTok{ }\KeywordTok{svm}\NormalTok{(ln_median_house_value }\OperatorTok{~}\NormalTok{. , }
                 \DataTypeTok{data =}\NormalTok{ dt.train, }
                 \DataTypeTok{type =} \StringTok{"eps-regression"}\NormalTok{, }
                 \DataTypeTok{kernel =} \StringTok{"radial"}\NormalTok{, }\DataTypeTok{cost =} \FloatTok{0.1}\NormalTok{, }\DataTypeTok{gamma =} \FloatTok{0.1}\NormalTok{)}
\NormalTok{svr_pred_}\DecValTok{7}\NormalTok{ <-}\StringTok{ }\KeywordTok{predict}\NormalTok{(svr_}\DecValTok{5}\NormalTok{, }\DataTypeTok{newdata =}\NormalTok{ dt.test)}
\NormalTok{rmse_svr_}\DecValTok{7}\NormalTok{ <-}\StringTok{ }\KeywordTok{sqrt}\NormalTok{(}\KeywordTok{MSE}\NormalTok{(}\DataTypeTok{y_pred =}\NormalTok{ svr_pred_}\DecValTok{7}\NormalTok{, }\DataTypeTok{y_true =}\NormalTok{ dt.test}\OperatorTok{$}\NormalTok{ln_median_house_value))}
\NormalTok{eps_reg[}\DecValTok{7}\NormalTok{] <-}\StringTok{ }\NormalTok{rmse_svr_}\DecValTok{7}
\NormalTok{rmse_svr_}\DecValTok{7}
\end{Highlighting}
\end{Shaded}

\begin{verbatim}
## [1] 0.430074
\end{verbatim}

\begin{Shaded}
\begin{Highlighting}[]
\NormalTok{svr_}\DecValTok{8}\NormalTok{ <-}\StringTok{ }\KeywordTok{svm}\NormalTok{(ln_median_house_value }\OperatorTok{~}\NormalTok{. , }
                 \DataTypeTok{data =}\NormalTok{ dt.train, }
                 \DataTypeTok{type =} \StringTok{"eps-regression"}\NormalTok{, }
                 \DataTypeTok{kernel =} \StringTok{"radial"}\NormalTok{, }\DataTypeTok{cost =} \DecValTok{1}\NormalTok{, }\DataTypeTok{gamma =} \FloatTok{0.1}\NormalTok{)}
\NormalTok{svr_pred_}\DecValTok{8}\NormalTok{ <-}\StringTok{ }\KeywordTok{predict}\NormalTok{(svr_}\DecValTok{5}\NormalTok{, }\DataTypeTok{newdata =}\NormalTok{ dt.test)}
\NormalTok{rmse_svr_}\DecValTok{8}\NormalTok{ <-}\StringTok{ }\KeywordTok{sqrt}\NormalTok{(}\KeywordTok{MSE}\NormalTok{(}\DataTypeTok{y_pred =}\NormalTok{ svr_pred_}\DecValTok{8}\NormalTok{, }\DataTypeTok{y_true =}\NormalTok{ dt.test}\OperatorTok{$}\NormalTok{ln_median_house_value))}
\NormalTok{eps_reg[}\DecValTok{8}\NormalTok{] <-}\StringTok{ }\NormalTok{rmse_svr_}\DecValTok{8}
\NormalTok{rmse_svr_}\DecValTok{8}
\end{Highlighting}
\end{Shaded}

\begin{verbatim}
## [1] 0.430074
\end{verbatim}

\hypertarget{kernel-polinomial-de-grado-2}{%
\subsubsection{Kernel polinomial de grado
2}\label{kernel-polinomial-de-grado-2}}

\begin{Shaded}
\begin{Highlighting}[]
\NormalTok{svr_}\DecValTok{9}\NormalTok{ <-}\StringTok{ }\KeywordTok{svm}\NormalTok{(ln_median_house_value }\OperatorTok{~}\NormalTok{. , }
                 \DataTypeTok{data =}\NormalTok{ dt.train, }
                 \DataTypeTok{type =} \StringTok{"eps-regression"}\NormalTok{, }
                 \DataTypeTok{kernel =} \StringTok{"polynomial"}\NormalTok{, }\DataTypeTok{cost =} \FloatTok{0.001}\NormalTok{, }\DataTypeTok{gamma =} \FloatTok{0.1}\NormalTok{, }\DataTypeTok{degree =} \DecValTok{2}\NormalTok{)}
\NormalTok{svr_pred_}\DecValTok{9}\NormalTok{ <-}\StringTok{ }\KeywordTok{predict}\NormalTok{(svr_}\DecValTok{5}\NormalTok{, }\DataTypeTok{newdata =}\NormalTok{ dt.test)}
\NormalTok{rmse_svr_}\DecValTok{9}\NormalTok{ <-}\StringTok{ }\KeywordTok{sqrt}\NormalTok{(}\KeywordTok{MSE}\NormalTok{(}\DataTypeTok{y_pred =}\NormalTok{ svr_pred_}\DecValTok{9}\NormalTok{, }\DataTypeTok{y_true =}\NormalTok{ dt.test}\OperatorTok{$}\NormalTok{ln_median_house_value))}
\NormalTok{eps_reg[}\DecValTok{9}\NormalTok{] <-}\StringTok{ }\NormalTok{rmse_svr_}\DecValTok{9}
\NormalTok{rmse_svr_}\DecValTok{9}
\end{Highlighting}
\end{Shaded}

\begin{verbatim}
## [1] 0.430074
\end{verbatim}

\begin{Shaded}
\begin{Highlighting}[]
\NormalTok{svr_}\DecValTok{10}\NormalTok{ <-}\StringTok{ }\KeywordTok{svm}\NormalTok{(ln_median_house_value }\OperatorTok{~}\NormalTok{. , }
                 \DataTypeTok{data =}\NormalTok{ dt.train, }
                 \DataTypeTok{type =} \StringTok{"eps-regression"}\NormalTok{, }
                 \DataTypeTok{kernel =} \StringTok{"polynomial"}\NormalTok{, }\DataTypeTok{cost =} \FloatTok{0.01}\NormalTok{, }\DataTypeTok{gamma =} \FloatTok{0.1}\NormalTok{, }\DataTypeTok{degree =} \DecValTok{2}\NormalTok{)}
\NormalTok{svr_pred_}\DecValTok{10}\NormalTok{ <-}\StringTok{ }\KeywordTok{predict}\NormalTok{(svr_}\DecValTok{5}\NormalTok{, }\DataTypeTok{newdata =}\NormalTok{ dt.test)}
\NormalTok{rmse_svr_}\DecValTok{10}\NormalTok{ <-}\StringTok{ }\KeywordTok{sqrt}\NormalTok{(}\KeywordTok{MSE}\NormalTok{(}\DataTypeTok{y_pred =}\NormalTok{ svr_pred_}\DecValTok{10}\NormalTok{, }\DataTypeTok{y_true =}\NormalTok{ dt.test}\OperatorTok{$}\NormalTok{ln_median_house_value))}
\NormalTok{eps_reg[}\DecValTok{10}\NormalTok{] <-}\StringTok{ }\NormalTok{rmse_svr_}\DecValTok{10}
\NormalTok{rmse_svr_}\DecValTok{10}
\end{Highlighting}
\end{Shaded}

\begin{verbatim}
## [1] 0.430074
\end{verbatim}

\begin{Shaded}
\begin{Highlighting}[]
\NormalTok{svr_}\DecValTok{11}\NormalTok{ <-}\StringTok{ }\KeywordTok{svm}\NormalTok{(ln_median_house_value }\OperatorTok{~}\NormalTok{. , }
                 \DataTypeTok{data =}\NormalTok{ dt.train, }
                 \DataTypeTok{type =} \StringTok{"eps-regression"}\NormalTok{, }
                 \DataTypeTok{kernel =} \StringTok{"polynomial"}\NormalTok{, }\DataTypeTok{cost =} \FloatTok{0.1}\NormalTok{, }\DataTypeTok{gamma =} \FloatTok{0.1}\NormalTok{, }\DataTypeTok{degree =} \DecValTok{2}\NormalTok{)}
\NormalTok{svr_pred_}\DecValTok{11}\NormalTok{ <-}\StringTok{ }\KeywordTok{predict}\NormalTok{(svr_}\DecValTok{5}\NormalTok{, }\DataTypeTok{newdata =}\NormalTok{ dt.test)}
\NormalTok{rmse_svr_}\DecValTok{11}\NormalTok{ <-}\StringTok{ }\KeywordTok{sqrt}\NormalTok{(}\KeywordTok{MSE}\NormalTok{(}\DataTypeTok{y_pred =}\NormalTok{ svr_pred_}\DecValTok{11}\NormalTok{, }\DataTypeTok{y_true =}\NormalTok{ dt.test}\OperatorTok{$}\NormalTok{ln_median_house_value))}
\NormalTok{eps_reg[}\DecValTok{11}\NormalTok{] <-}\StringTok{ }\NormalTok{rmse_svr_}\DecValTok{11}
\NormalTok{rmse_svr_}\DecValTok{11}
\end{Highlighting}
\end{Shaded}

\begin{verbatim}
## [1] 0.430074
\end{verbatim}

\begin{Shaded}
\begin{Highlighting}[]
\NormalTok{svr_}\DecValTok{12}\NormalTok{ <-}\StringTok{ }\KeywordTok{svm}\NormalTok{(ln_median_house_value }\OperatorTok{~}\NormalTok{. , }
                 \DataTypeTok{data =}\NormalTok{ dt.train, }
                 \DataTypeTok{type =} \StringTok{"eps-regression"}\NormalTok{, }
                 \DataTypeTok{kernel =} \StringTok{"polynomial"}\NormalTok{, }\DataTypeTok{cost =} \DecValTok{1}\NormalTok{, }\DataTypeTok{gamma =} \FloatTok{0.1}\NormalTok{, }\DataTypeTok{degree =} \DecValTok{2}\NormalTok{)}
\NormalTok{svr_pred_}\DecValTok{12}\NormalTok{ <-}\StringTok{ }\KeywordTok{predict}\NormalTok{(svr_}\DecValTok{5}\NormalTok{, }\DataTypeTok{newdata =}\NormalTok{ dt.test)}
\NormalTok{rmse_svr_}\DecValTok{12}\NormalTok{ <-}\StringTok{ }\KeywordTok{sqrt}\NormalTok{(}\KeywordTok{MSE}\NormalTok{(}\DataTypeTok{y_pred =}\NormalTok{ svr_pred_}\DecValTok{12}\NormalTok{, }\DataTypeTok{y_true =}\NormalTok{ dt.test}\OperatorTok{$}\NormalTok{ln_median_house_value))}
\NormalTok{eps_reg[}\DecValTok{12}\NormalTok{] <-}\StringTok{ }\NormalTok{rmse_svr_}\DecValTok{12}
\NormalTok{rmse_svr_}\DecValTok{12}
\end{Highlighting}
\end{Shaded}

\begin{verbatim}
## [1] 0.430074
\end{verbatim}

\hypertarget{kernel-polinomial-de-grado-3}{%
\subsubsection{Kernel polinomial de grado
3}\label{kernel-polinomial-de-grado-3}}

\begin{Shaded}
\begin{Highlighting}[]
\NormalTok{svr_}\DecValTok{13}\NormalTok{ <-}\StringTok{ }\KeywordTok{svm}\NormalTok{(ln_median_house_value }\OperatorTok{~}\NormalTok{. , }
                 \DataTypeTok{data =}\NormalTok{ dt.train, }
                 \DataTypeTok{type =} \StringTok{"eps-regression"}\NormalTok{, }
                 \DataTypeTok{kernel =} \StringTok{"polynomial"}\NormalTok{, }\DataTypeTok{cost =} \FloatTok{0.001}\NormalTok{, }\DataTypeTok{gamma =} \FloatTok{0.1}\NormalTok{, }\DataTypeTok{degree =} \DecValTok{3}\NormalTok{)}
\NormalTok{svr_pred_}\DecValTok{13}\NormalTok{ <-}\StringTok{ }\KeywordTok{predict}\NormalTok{(svr_}\DecValTok{5}\NormalTok{, }\DataTypeTok{newdata =}\NormalTok{ dt.test)}
\NormalTok{rmse_svr_}\DecValTok{13}\NormalTok{ <-}\StringTok{ }\KeywordTok{sqrt}\NormalTok{(}\KeywordTok{MSE}\NormalTok{(}\DataTypeTok{y_pred =}\NormalTok{ svr_pred_}\DecValTok{13}\NormalTok{, }\DataTypeTok{y_true =}\NormalTok{ dt.test}\OperatorTok{$}\NormalTok{ln_median_house_value))}
\NormalTok{eps_reg[}\DecValTok{13}\NormalTok{] <-}\StringTok{ }\NormalTok{rmse_svr_}\DecValTok{13}
\NormalTok{rmse_svr_}\DecValTok{13}
\end{Highlighting}
\end{Shaded}

\begin{verbatim}
## [1] 0.430074
\end{verbatim}

\begin{Shaded}
\begin{Highlighting}[]
\NormalTok{svr_}\DecValTok{14}\NormalTok{ <-}\StringTok{ }\KeywordTok{svm}\NormalTok{(ln_median_house_value }\OperatorTok{~}\NormalTok{. , }
                 \DataTypeTok{data =}\NormalTok{ dt.train, }
                 \DataTypeTok{type =} \StringTok{"eps-regression"}\NormalTok{, }
                 \DataTypeTok{kernel =} \StringTok{"polynomial"}\NormalTok{, }\DataTypeTok{cost =} \FloatTok{0.01}\NormalTok{, }\DataTypeTok{gamma =} \FloatTok{0.1}\NormalTok{, }\DataTypeTok{degree =} \DecValTok{3}\NormalTok{)}
\NormalTok{svr_pred_}\DecValTok{14}\NormalTok{ <-}\StringTok{ }\KeywordTok{predict}\NormalTok{(svr_}\DecValTok{5}\NormalTok{, }\DataTypeTok{newdata =}\NormalTok{ dt.test)}
\NormalTok{rmse_svr_}\DecValTok{14}\NormalTok{ <-}\StringTok{ }\KeywordTok{sqrt}\NormalTok{(}\KeywordTok{MSE}\NormalTok{(}\DataTypeTok{y_pred =}\NormalTok{ svr_pred_}\DecValTok{14}\NormalTok{, }\DataTypeTok{y_true =}\NormalTok{ dt.test}\OperatorTok{$}\NormalTok{ln_median_house_value))}
\NormalTok{eps_reg[}\DecValTok{14}\NormalTok{] <-}\StringTok{ }\NormalTok{rmse_svr_}\DecValTok{14}
\NormalTok{rmse_svr_}\DecValTok{14}
\end{Highlighting}
\end{Shaded}

\begin{verbatim}
## [1] 0.430074
\end{verbatim}

\begin{Shaded}
\begin{Highlighting}[]
\NormalTok{svr_}\DecValTok{15}\NormalTok{ <-}\StringTok{ }\KeywordTok{svm}\NormalTok{(ln_median_house_value }\OperatorTok{~}\NormalTok{. , }
                 \DataTypeTok{data =}\NormalTok{ dt.train, }
                 \DataTypeTok{type =} \StringTok{"eps-regression"}\NormalTok{, }
                 \DataTypeTok{kernel =} \StringTok{"polynomial"}\NormalTok{, }\DataTypeTok{cost =} \FloatTok{0.1}\NormalTok{, }\DataTypeTok{gamma =} \FloatTok{0.1}\NormalTok{, }\DataTypeTok{degree =} \DecValTok{3}\NormalTok{)}
\NormalTok{svr_pred_}\DecValTok{15}\NormalTok{ <-}\StringTok{ }\KeywordTok{predict}\NormalTok{(svr_}\DecValTok{5}\NormalTok{, }\DataTypeTok{newdata =}\NormalTok{ dt.test)}
\NormalTok{rmse_svr_}\DecValTok{15}\NormalTok{ <-}\StringTok{ }\KeywordTok{sqrt}\NormalTok{(}\KeywordTok{MSE}\NormalTok{(}\DataTypeTok{y_pred =}\NormalTok{ svr_pred_}\DecValTok{15}\NormalTok{, }\DataTypeTok{y_true =}\NormalTok{ dt.test}\OperatorTok{$}\NormalTok{ln_median_house_value))}
\NormalTok{eps_reg[}\DecValTok{15}\NormalTok{] <-}\StringTok{ }\NormalTok{rmse_svr_}\DecValTok{15}
\NormalTok{rmse_svr_}\DecValTok{15}
\end{Highlighting}
\end{Shaded}

\begin{verbatim}
## [1] 0.430074
\end{verbatim}

\begin{Shaded}
\begin{Highlighting}[]
\NormalTok{svr_}\DecValTok{16}\NormalTok{ <-}\StringTok{ }\KeywordTok{svm}\NormalTok{(ln_median_house_value }\OperatorTok{~}\NormalTok{. , }
                 \DataTypeTok{data =}\NormalTok{ dt.train, }
                 \DataTypeTok{type =} \StringTok{"eps-regression"}\NormalTok{, }
                 \DataTypeTok{kernel =} \StringTok{"polynomial"}\NormalTok{, }\DataTypeTok{cost =} \DecValTok{1}\NormalTok{, }\DataTypeTok{gamma =} \FloatTok{0.1}\NormalTok{, }\DataTypeTok{degree =} \DecValTok{3}\NormalTok{)}
\NormalTok{svr_pred_}\DecValTok{16}\NormalTok{ <-}\StringTok{ }\KeywordTok{predict}\NormalTok{(svr_}\DecValTok{5}\NormalTok{, }\DataTypeTok{newdata =}\NormalTok{ dt.test)}
\NormalTok{rmse_svr_}\DecValTok{16}\NormalTok{ <-}\StringTok{ }\KeywordTok{sqrt}\NormalTok{(}\KeywordTok{MSE}\NormalTok{(}\DataTypeTok{y_pred =}\NormalTok{ svr_pred_}\DecValTok{13}\NormalTok{, }\DataTypeTok{y_true =}\NormalTok{ dt.test}\OperatorTok{$}\NormalTok{ln_median_house_value))}
\NormalTok{eps_reg[}\DecValTok{16}\NormalTok{] <-}\StringTok{ }\NormalTok{rmse_svr_}\DecValTok{16}
\NormalTok{rmse_svr_}\DecValTok{16}
\end{Highlighting}
\end{Shaded}

\begin{verbatim}
## [1] 0.430074
\end{verbatim}

Visualizamos los errores que hemos obtenido con las Epsilon-SVR.
Recordemos que hay cuatro modelos con costes diferentes para cada tipo
de kernel.

\begin{Shaded}
\begin{Highlighting}[]
\NormalTok{eps_reg}
\end{Highlighting}
\end{Shaded}

\begin{verbatim}
##  [1] 0.3643270 0.3553593 0.3562273 0.3564667 0.4300740 0.4300740 0.4300740
##  [8] 0.4300740 0.4300740 0.4300740 0.4300740 0.4300740 0.4300740 0.4300740
## [15] 0.4300740 0.4300740
\end{verbatim}

Vemos que los únicos modelos que varían los errores son los SVR con
kernel lineal, el resto tienen el mismo error. Vamos a coger el modelo
con kernel lineal que tiene el error mínimo y mostrar los coeficientes
obtenidos. Para el resto de modelos, como tienen el mismo error
cogeremos uno de cada tipo y mostramos los coeficientes. Se muestra
primero los coeficientes y el modelo nos proporciona el intercepto en
los modelos que no tienen kernel lineal, para los modelos con kernel
lineal utilizamos la función \texttt{coef}.

\begin{Shaded}
\begin{Highlighting}[]
\KeywordTok{coef}\NormalTok{(svr_}\DecValTok{2}\NormalTok{)}
\end{Highlighting}
\end{Shaded}

\begin{verbatim}
##                  (Intercept)                median_income 
##                 -0.002033054                  1.178249585 
##              median_income_2              median_income_3 
##                  0.128703279                 -0.414484252 
##        ln_housing_median_age    ln_total_rooms_population 
##                  0.175245047                 -0.583977808 
## ln_total_bedrooms_population     ln_population_households 
##                  0.490559146                 -0.191890079 
##                ln_households 
##                  0.085330259
\end{verbatim}

Las características que influyen de forma positiva en el precio medio de
la vivienda son las que tienen coeficiente positivo y viceversa. Vemos
que este modelo nos refleja características que influyen en la subida
del precio medio de la vivienda como el ingreso medio, la edad media de
las viviendas, etc.

\begin{Shaded}
\begin{Highlighting}[]
\KeywordTok{t}\NormalTok{(svr_}\DecValTok{5}\OperatorTok{$}\NormalTok{coefs) }\OperatorTok\StringTok{ }\NormalTok{svr_}\DecValTok{5}\OperatorTok{$}\NormalTok{SV}
\end{Highlighting}
\end{Shaded}

\begin{verbatim}
##      median_income median_income_2 median_income_3 ln_housing_median_age
## [1,]      5.141163        4.631986        3.955364              1.119932
##      ln_total_rooms_population ln_total_bedrooms_population
## [1,]                  1.452933                    0.8935851
##      ln_population_households ln_households
## [1,]                 -1.86126      1.263571
\end{verbatim}

\begin{Shaded}
\begin{Highlighting}[]
\NormalTok{svr_}\DecValTok{5}\OperatorTok{$}\NormalTok{rho}
\end{Highlighting}
\end{Shaded}

\begin{verbatim}
## [1] -0.2285464
\end{verbatim}

En este modelo la única variable que influye de manera negativa en el
precio medio de las viviendas es \texttt{ln\_population\_households}.

\begin{Shaded}
\begin{Highlighting}[]
\KeywordTok{t}\NormalTok{(svr_}\DecValTok{9}\OperatorTok{$}\NormalTok{coefs) }\OperatorTok\StringTok{ }\NormalTok{svr_}\DecValTok{9}\OperatorTok{$}\NormalTok{SV}
\end{Highlighting}
\end{Shaded}

\begin{verbatim}
##      median_income median_income_2 median_income_3 ln_housing_median_age
## [1,]       6.83127        5.387778        3.708434             0.8749337
##      ln_total_rooms_population ln_total_bedrooms_population
## [1,]                  3.591677                     1.855682
##      ln_population_households ln_households
## [1,]                -2.800658      1.363209
\end{verbatim}

\begin{Shaded}
\begin{Highlighting}[]
\NormalTok{svr_}\DecValTok{9}\OperatorTok{$}\NormalTok{rho}
\end{Highlighting}
\end{Shaded}

\begin{verbatim}
## [1] -0.007575784
\end{verbatim}

\begin{Shaded}
\begin{Highlighting}[]
\KeywordTok{t}\NormalTok{(svr_}\DecValTok{13}\OperatorTok{$}\NormalTok{coefs) }\OperatorTok\StringTok{ }\NormalTok{svr_}\DecValTok{13}\OperatorTok{$}\NormalTok{SV}
\end{Highlighting}
\end{Shaded}

\begin{verbatim}
##      median_income median_income_2 median_income_3 ln_housing_median_age
## [1,]      5.780997         4.53874        3.092505              1.247389
##      ln_total_rooms_population ln_total_bedrooms_population
## [1,]                  2.625476                     1.559946
##      ln_population_households ln_households
## [1,]                -2.655351      1.151274
\end{verbatim}

\begin{Shaded}
\begin{Highlighting}[]
\NormalTok{svr_}\DecValTok{13}\OperatorTok{$}\NormalTok{rho}
\end{Highlighting}
\end{Shaded}

\begin{verbatim}
## [1] -0.03050197
\end{verbatim}

Los dos últimos modelos son iguales que el segundo: la única variable
que tiene coeficiente negativo es \texttt{ln\_population\_households}.

\hypertarget{nu-svr}{%
\subsection{Nu-SVR}\label{nu-svr}}

\hypertarget{kernel-lineal-1}{%
\subsubsection{Kernel lineal}\label{kernel-lineal-1}}

\begin{Shaded}
\begin{Highlighting}[]
\NormalTok{svr_}\DecValTok{1}\NormalTok{ <-}\StringTok{ }\KeywordTok{svm}\NormalTok{(ln_median_house_value }\OperatorTok{~}\NormalTok{. , }
                 \DataTypeTok{data =}\NormalTok{ dt.train, }
                 \DataTypeTok{type =} \StringTok{"nu-regression"}\NormalTok{, }
                 \DataTypeTok{kernel =} \StringTok{"linear"}\NormalTok{, }\DataTypeTok{cost =} \FloatTok{0.001}\NormalTok{)}
\NormalTok{svr_pred_}\DecValTok{1}\NormalTok{ <-}\StringTok{ }\KeywordTok{predict}\NormalTok{(svr_}\DecValTok{1}\NormalTok{, }\DataTypeTok{newdata =}\NormalTok{ dt.test)}
\NormalTok{rmse_svr_}\DecValTok{1}\NormalTok{ <-}\StringTok{ }\KeywordTok{sqrt}\NormalTok{(}\KeywordTok{MSE}\NormalTok{(}\DataTypeTok{y_pred =}\NormalTok{ svr_pred_}\DecValTok{1}\NormalTok{, }\DataTypeTok{y_true =}\NormalTok{ dt.test}\OperatorTok{$}\NormalTok{ln_median_house_value))}
\NormalTok{nu_reg[}\DecValTok{1}\NormalTok{] <-}\StringTok{ }\NormalTok{rmse_svr_}\DecValTok{1}
\NormalTok{rmse_svr_}\DecValTok{1}
\end{Highlighting}
\end{Shaded}

\begin{verbatim}
## [1] 0.3675666
\end{verbatim}

\begin{Shaded}
\begin{Highlighting}[]
\NormalTok{svr_}\DecValTok{2}\NormalTok{ <-}\StringTok{ }\KeywordTok{svm}\NormalTok{(ln_median_house_value }\OperatorTok{~}\NormalTok{. , }
                 \DataTypeTok{data =}\NormalTok{ dt.train, }
                 \DataTypeTok{type =} \StringTok{"nu-regression"}\NormalTok{, }
                 \DataTypeTok{kernel =} \StringTok{"linear"}\NormalTok{, }\DataTypeTok{cost =} \FloatTok{0.01}\NormalTok{)}
\NormalTok{svr_pred_}\DecValTok{2}\NormalTok{ <-}\StringTok{ }\KeywordTok{predict}\NormalTok{(svr_}\DecValTok{2}\NormalTok{, }\DataTypeTok{newdata =}\NormalTok{ dt.test)}
\NormalTok{rmse_svr_}\DecValTok{2}\NormalTok{ <-}\StringTok{ }\KeywordTok{sqrt}\NormalTok{(}\KeywordTok{MSE}\NormalTok{(}\DataTypeTok{y_pred =}\NormalTok{ svr_pred_}\DecValTok{2}\NormalTok{, }\DataTypeTok{y_true =}\NormalTok{ dt.test}\OperatorTok{$}\NormalTok{ln_median_house_value))}
\NormalTok{nu_reg[}\DecValTok{2}\NormalTok{] <-}\StringTok{ }\NormalTok{rmse_svr_}\DecValTok{2}
\NormalTok{rmse_svr_}\DecValTok{2}
\end{Highlighting}
\end{Shaded}

\begin{verbatim}
## [1] 0.355153
\end{verbatim}

\begin{Shaded}
\begin{Highlighting}[]
\NormalTok{svr_}\DecValTok{3}\NormalTok{ <-}\StringTok{ }\KeywordTok{svm}\NormalTok{(ln_median_house_value }\OperatorTok{~}\NormalTok{. , }
                 \DataTypeTok{data =}\NormalTok{ dt.train, }
                 \DataTypeTok{type =} \StringTok{"nu-regression"}\NormalTok{, }
                 \DataTypeTok{kernel =} \StringTok{"linear"}\NormalTok{, }\DataTypeTok{cost =} \FloatTok{0.1}\NormalTok{)}
\NormalTok{svr_pred_}\DecValTok{3}\NormalTok{ <-}\StringTok{ }\KeywordTok{predict}\NormalTok{(svr_}\DecValTok{3}\NormalTok{, }\DataTypeTok{newdata =}\NormalTok{ dt.test)}
\NormalTok{rmse_svr_}\DecValTok{3}\NormalTok{ <-}\StringTok{ }\KeywordTok{sqrt}\NormalTok{(}\KeywordTok{MSE}\NormalTok{(}\DataTypeTok{y_pred =}\NormalTok{ svr_pred_}\DecValTok{3}\NormalTok{, }\DataTypeTok{y_true =}\NormalTok{ dt.test}\OperatorTok{$}\NormalTok{ln_median_house_value))}
\NormalTok{nu_reg[}\DecValTok{3}\NormalTok{] <-}\StringTok{ }\NormalTok{rmse_svr_}\DecValTok{3}
\NormalTok{rmse_svr_}\DecValTok{3}
\end{Highlighting}
\end{Shaded}

\begin{verbatim}
## [1] 0.3561244
\end{verbatim}

\begin{Shaded}
\begin{Highlighting}[]
\NormalTok{svr_}\DecValTok{4}\NormalTok{ <-}\StringTok{ }\KeywordTok{svm}\NormalTok{(ln_median_house_value }\OperatorTok{~}\NormalTok{. , }
                 \DataTypeTok{data =}\NormalTok{ dt.train, }
                 \DataTypeTok{type =} \StringTok{"nu-regression"}\NormalTok{, }
                 \DataTypeTok{kernel =} \StringTok{"linear"}\NormalTok{, }\DataTypeTok{cost =} \DecValTok{1}\NormalTok{)}
\NormalTok{svr_pred_}\DecValTok{4}\NormalTok{ <-}\StringTok{ }\KeywordTok{predict}\NormalTok{(svr_}\DecValTok{4}\NormalTok{, }\DataTypeTok{newdata =}\NormalTok{ dt.test)}
\NormalTok{rmse_svr_}\DecValTok{4}\NormalTok{ <-}\StringTok{ }\KeywordTok{sqrt}\NormalTok{(}\KeywordTok{MSE}\NormalTok{(}\DataTypeTok{y_pred =}\NormalTok{ svr_pred_}\DecValTok{4}\NormalTok{, }\DataTypeTok{y_true =}\NormalTok{ dt.test}\OperatorTok{$}\NormalTok{ln_median_house_value))}
\NormalTok{nu_reg[}\DecValTok{4}\NormalTok{] <-}\StringTok{ }\NormalTok{rmse_svr_}\DecValTok{4}
\NormalTok{rmse_svr_}\DecValTok{4}
\end{Highlighting}
\end{Shaded}

\begin{verbatim}
## [1] 0.3564781
\end{verbatim}

\hypertarget{kernel-radial-1}{%
\subsubsection{Kernel radial}\label{kernel-radial-1}}

\begin{Shaded}
\begin{Highlighting}[]
\NormalTok{svr_}\DecValTok{5}\NormalTok{ <-}\StringTok{ }\KeywordTok{svm}\NormalTok{(ln_median_house_value }\OperatorTok{~}\NormalTok{. , }
                 \DataTypeTok{data =}\NormalTok{ dt.train, }
                 \DataTypeTok{type =} \StringTok{"nu-regression"}\NormalTok{, }
                 \DataTypeTok{kernel =} \StringTok{"radial"}\NormalTok{, }\DataTypeTok{cost =} \FloatTok{0.001}\NormalTok{, }\DataTypeTok{gamma =} \FloatTok{0.1}\NormalTok{)}
\NormalTok{svr_pred_}\DecValTok{5}\NormalTok{ <-}\StringTok{ }\KeywordTok{predict}\NormalTok{(svr_}\DecValTok{5}\NormalTok{, }\DataTypeTok{newdata =}\NormalTok{ dt.test)}
\NormalTok{rmse_svr_}\DecValTok{5}\NormalTok{ <-}\StringTok{ }\KeywordTok{sqrt}\NormalTok{(}\KeywordTok{MSE}\NormalTok{(}\DataTypeTok{y_pred =}\NormalTok{ svr_pred_}\DecValTok{5}\NormalTok{, }\DataTypeTok{y_true =}\NormalTok{ dt.test}\OperatorTok{$}\NormalTok{ln_median_house_value))}
\NormalTok{nu_reg[}\DecValTok{5}\NormalTok{] <-}\StringTok{ }\NormalTok{rmse_svr_}\DecValTok{5}
\NormalTok{rmse_svr_}\DecValTok{5}
\end{Highlighting}
\end{Shaded}

\begin{verbatim}
## [1] 0.4395939
\end{verbatim}

\begin{Shaded}
\begin{Highlighting}[]
\NormalTok{svr_}\DecValTok{6}\NormalTok{ <-}\StringTok{ }\KeywordTok{svm}\NormalTok{(ln_median_house_value }\OperatorTok{~}\NormalTok{. , }
                 \DataTypeTok{data =}\NormalTok{ dt.train, }
                 \DataTypeTok{type =} \StringTok{"nu-regression"}\NormalTok{, }
                 \DataTypeTok{kernel =} \StringTok{"radial"}\NormalTok{, }\DataTypeTok{cost =} \FloatTok{0.01}\NormalTok{, }\DataTypeTok{gamma =} \FloatTok{0.1}\NormalTok{)}
\NormalTok{svr_pred_}\DecValTok{6}\NormalTok{ <-}\StringTok{ }\KeywordTok{predict}\NormalTok{(svr_}\DecValTok{5}\NormalTok{, }\DataTypeTok{newdata =}\NormalTok{ dt.test)}
\NormalTok{rmse_svr_}\DecValTok{6}\NormalTok{ <-}\StringTok{ }\KeywordTok{sqrt}\NormalTok{(}\KeywordTok{MSE}\NormalTok{(}\DataTypeTok{y_pred =}\NormalTok{ svr_pred_}\DecValTok{6}\NormalTok{, }\DataTypeTok{y_true =}\NormalTok{ dt.test}\OperatorTok{$}\NormalTok{ln_median_house_value))}
\NormalTok{nu_reg[}\DecValTok{6}\NormalTok{] <-}\StringTok{ }\NormalTok{rmse_svr_}\DecValTok{6}
\NormalTok{rmse_svr_}\DecValTok{6}
\end{Highlighting}
\end{Shaded}

\begin{verbatim}
## [1] 0.4395939
\end{verbatim}

\begin{Shaded}
\begin{Highlighting}[]
\NormalTok{svr_}\DecValTok{7}\NormalTok{ <-}\StringTok{ }\KeywordTok{svm}\NormalTok{(ln_median_house_value }\OperatorTok{~}\NormalTok{. , }
                 \DataTypeTok{data =}\NormalTok{ dt.train, }
                 \DataTypeTok{type =} \StringTok{"nu-regression"}\NormalTok{, }
                 \DataTypeTok{kernel =} \StringTok{"radial"}\NormalTok{, }\DataTypeTok{cost =} \FloatTok{0.1}\NormalTok{, }\DataTypeTok{gamma =} \FloatTok{0.1}\NormalTok{)}
\NormalTok{svr_pred_}\DecValTok{7}\NormalTok{ <-}\StringTok{ }\KeywordTok{predict}\NormalTok{(svr_}\DecValTok{5}\NormalTok{, }\DataTypeTok{newdata =}\NormalTok{ dt.test)}
\NormalTok{rmse_svr_}\DecValTok{7}\NormalTok{ <-}\StringTok{ }\KeywordTok{sqrt}\NormalTok{(}\KeywordTok{MSE}\NormalTok{(}\DataTypeTok{y_pred =}\NormalTok{ svr_pred_}\DecValTok{7}\NormalTok{, }\DataTypeTok{y_true =}\NormalTok{ dt.test}\OperatorTok{$}\NormalTok{ln_median_house_value))}
\NormalTok{nu_reg[}\DecValTok{7}\NormalTok{] <-}\StringTok{ }\NormalTok{rmse_svr_}\DecValTok{7}
\NormalTok{rmse_svr_}\DecValTok{7}
\end{Highlighting}
\end{Shaded}

\begin{verbatim}
## [1] 0.4395939
\end{verbatim}

\begin{Shaded}
\begin{Highlighting}[]
\NormalTok{svr_}\DecValTok{8}\NormalTok{ <-}\StringTok{ }\KeywordTok{svm}\NormalTok{(ln_median_house_value }\OperatorTok{~}\NormalTok{. , }
                 \DataTypeTok{data =}\NormalTok{ dt.train, }
                 \DataTypeTok{type =} \StringTok{"nu-regression"}\NormalTok{, }
                 \DataTypeTok{kernel =} \StringTok{"radial"}\NormalTok{, }\DataTypeTok{cost =} \DecValTok{1}\NormalTok{, }\DataTypeTok{gamma =} \FloatTok{0.1}\NormalTok{)}
\NormalTok{svr_pred_}\DecValTok{8}\NormalTok{ <-}\StringTok{ }\KeywordTok{predict}\NormalTok{(svr_}\DecValTok{5}\NormalTok{, }\DataTypeTok{newdata =}\NormalTok{ dt.test)}
\NormalTok{rmse_svr_}\DecValTok{8}\NormalTok{ <-}\StringTok{ }\KeywordTok{sqrt}\NormalTok{(}\KeywordTok{MSE}\NormalTok{(}\DataTypeTok{y_pred =}\NormalTok{ svr_pred_}\DecValTok{8}\NormalTok{, }\DataTypeTok{y_true =}\NormalTok{ dt.test}\OperatorTok{$}\NormalTok{ln_median_house_value))}
\NormalTok{nu_reg[}\DecValTok{8}\NormalTok{] <-}\StringTok{ }\NormalTok{rmse_svr_}\DecValTok{8}
\NormalTok{rmse_svr_}\DecValTok{8}
\end{Highlighting}
\end{Shaded}

\begin{verbatim}
## [1] 0.4395939
\end{verbatim}

\hypertarget{kernel-polinomial-de-grado-2-1}{%
\subsubsection{Kernel polinomial de grado
2}\label{kernel-polinomial-de-grado-2-1}}

\begin{Shaded}
\begin{Highlighting}[]
\NormalTok{svr_}\DecValTok{9}\NormalTok{ <-}\StringTok{ }\KeywordTok{svm}\NormalTok{(ln_median_house_value }\OperatorTok{~}\NormalTok{. , }
                 \DataTypeTok{data =}\NormalTok{ dt.train, }
                 \DataTypeTok{type =} \StringTok{"nu-regression"}\NormalTok{, }
                 \DataTypeTok{kernel =} \StringTok{"polynomial"}\NormalTok{, }\DataTypeTok{cost =} \FloatTok{0.001}\NormalTok{, }\DataTypeTok{gamma =} \FloatTok{0.1}\NormalTok{, }\DataTypeTok{degree =} \DecValTok{2}\NormalTok{)}
\NormalTok{svr_pred_}\DecValTok{9}\NormalTok{ <-}\StringTok{ }\KeywordTok{predict}\NormalTok{(svr_}\DecValTok{5}\NormalTok{, }\DataTypeTok{newdata =}\NormalTok{ dt.test)}
\NormalTok{rmse_svr_}\DecValTok{9}\NormalTok{ <-}\StringTok{ }\KeywordTok{sqrt}\NormalTok{(}\KeywordTok{MSE}\NormalTok{(}\DataTypeTok{y_pred =}\NormalTok{ svr_pred_}\DecValTok{9}\NormalTok{, }\DataTypeTok{y_true =}\NormalTok{ dt.test}\OperatorTok{$}\NormalTok{ln_median_house_value))}
\NormalTok{nu_reg[}\DecValTok{9}\NormalTok{] <-}\StringTok{ }\NormalTok{rmse_svr_}\DecValTok{9}
\NormalTok{rmse_svr_}\DecValTok{9}
\end{Highlighting}
\end{Shaded}

\begin{verbatim}
## [1] 0.4395939
\end{verbatim}

\begin{Shaded}
\begin{Highlighting}[]
\NormalTok{svr_}\DecValTok{10}\NormalTok{ <-}\StringTok{ }\KeywordTok{svm}\NormalTok{(ln_median_house_value }\OperatorTok{~}\NormalTok{. , }
                 \DataTypeTok{data =}\NormalTok{ dt.train, }
                 \DataTypeTok{type =} \StringTok{"nu-regression"}\NormalTok{, }
                 \DataTypeTok{kernel =} \StringTok{"polynomial"}\NormalTok{, }\DataTypeTok{cost =} \FloatTok{0.01}\NormalTok{, }\DataTypeTok{gamma =} \FloatTok{0.1}\NormalTok{, }\DataTypeTok{degree =} \DecValTok{2}\NormalTok{)}
\NormalTok{svr_pred_}\DecValTok{10}\NormalTok{ <-}\StringTok{ }\KeywordTok{predict}\NormalTok{(svr_}\DecValTok{5}\NormalTok{, }\DataTypeTok{newdata =}\NormalTok{ dt.test)}
\NormalTok{rmse_svr_}\DecValTok{10}\NormalTok{ <-}\StringTok{ }\KeywordTok{sqrt}\NormalTok{(}\KeywordTok{MSE}\NormalTok{(}\DataTypeTok{y_pred =}\NormalTok{ svr_pred_}\DecValTok{10}\NormalTok{, }\DataTypeTok{y_true =}\NormalTok{ dt.test}\OperatorTok{$}\NormalTok{ln_median_house_value))}
\NormalTok{nu_reg[}\DecValTok{10}\NormalTok{] <-}\StringTok{ }\NormalTok{rmse_svr_}\DecValTok{10}
\NormalTok{rmse_svr_}\DecValTok{10}
\end{Highlighting}
\end{Shaded}

\begin{verbatim}
## [1] 0.4395939
\end{verbatim}

\begin{Shaded}
\begin{Highlighting}[]
\NormalTok{svr_}\DecValTok{11}\NormalTok{ <-}\StringTok{ }\KeywordTok{svm}\NormalTok{(ln_median_house_value }\OperatorTok{~}\NormalTok{. , }
                 \DataTypeTok{data =}\NormalTok{ dt.train, }
                 \DataTypeTok{type =} \StringTok{"nu-regression"}\NormalTok{, }
                 \DataTypeTok{kernel =} \StringTok{"polynomial"}\NormalTok{, }\DataTypeTok{cost =} \FloatTok{0.1}\NormalTok{, }\DataTypeTok{gamma =} \FloatTok{0.1}\NormalTok{, }\DataTypeTok{degree =} \DecValTok{2}\NormalTok{)}
\NormalTok{svr_pred_}\DecValTok{11}\NormalTok{ <-}\StringTok{ }\KeywordTok{predict}\NormalTok{(svr_}\DecValTok{5}\NormalTok{, }\DataTypeTok{newdata =}\NormalTok{ dt.test)}
\NormalTok{rmse_svr_}\DecValTok{11}\NormalTok{ <-}\StringTok{ }\KeywordTok{sqrt}\NormalTok{(}\KeywordTok{MSE}\NormalTok{(}\DataTypeTok{y_pred =}\NormalTok{ svr_pred_}\DecValTok{11}\NormalTok{, }\DataTypeTok{y_true =}\NormalTok{ dt.test}\OperatorTok{$}\NormalTok{ln_median_house_value))}
\NormalTok{nu_reg[}\DecValTok{11}\NormalTok{] <-}\StringTok{ }\NormalTok{rmse_svr_}\DecValTok{11}
\NormalTok{rmse_svr_}\DecValTok{11}
\end{Highlighting}
\end{Shaded}

\begin{verbatim}
## [1] 0.4395939
\end{verbatim}

\begin{Shaded}
\begin{Highlighting}[]
\NormalTok{svr_}\DecValTok{12}\NormalTok{ <-}\StringTok{ }\KeywordTok{svm}\NormalTok{(ln_median_house_value }\OperatorTok{~}\NormalTok{. , }
                 \DataTypeTok{data =}\NormalTok{ dt.train, }
                 \DataTypeTok{type =} \StringTok{"nu-regression"}\NormalTok{, }
                 \DataTypeTok{kernel =} \StringTok{"polynomial"}\NormalTok{, }\DataTypeTok{cost =} \DecValTok{1}\NormalTok{, }\DataTypeTok{gamma =} \FloatTok{0.1}\NormalTok{, }\DataTypeTok{degree =} \DecValTok{2}\NormalTok{)}
\NormalTok{svr_pred_}\DecValTok{12}\NormalTok{ <-}\StringTok{ }\KeywordTok{predict}\NormalTok{(svr_}\DecValTok{5}\NormalTok{, }\DataTypeTok{newdata =}\NormalTok{ dt.test)}
\NormalTok{rmse_svr_}\DecValTok{12}\NormalTok{ <-}\StringTok{ }\KeywordTok{sqrt}\NormalTok{(}\KeywordTok{MSE}\NormalTok{(}\DataTypeTok{y_pred =}\NormalTok{ svr_pred_}\DecValTok{12}\NormalTok{, }\DataTypeTok{y_true =}\NormalTok{ dt.test}\OperatorTok{$}\NormalTok{ln_median_house_value))}
\NormalTok{nu_reg[}\DecValTok{12}\NormalTok{] <-}\StringTok{ }\NormalTok{rmse_svr_}\DecValTok{12}
\NormalTok{rmse_svr_}\DecValTok{12}
\end{Highlighting}
\end{Shaded}

\begin{verbatim}
## [1] 0.4395939
\end{verbatim}

\hypertarget{kernel-polinomial-de-grado-3-1}{%
\subsubsection{Kernel polinomial de grado
3}\label{kernel-polinomial-de-grado-3-1}}

\begin{Shaded}
\begin{Highlighting}[]
\NormalTok{svr_}\DecValTok{13}\NormalTok{ <-}\StringTok{ }\KeywordTok{svm}\NormalTok{(ln_median_house_value }\OperatorTok{~}\NormalTok{. , }
                 \DataTypeTok{data =}\NormalTok{ dt.train, }
                 \DataTypeTok{type =} \StringTok{"nu-regression"}\NormalTok{, }
                 \DataTypeTok{kernel =} \StringTok{"polynomial"}\NormalTok{, }\DataTypeTok{cost =} \FloatTok{0.001}\NormalTok{, }\DataTypeTok{gamma =} \FloatTok{0.1}\NormalTok{, }\DataTypeTok{degree =} \DecValTok{3}\NormalTok{)}
\NormalTok{svr_pred_}\DecValTok{13}\NormalTok{ <-}\StringTok{ }\KeywordTok{predict}\NormalTok{(svr_}\DecValTok{5}\NormalTok{, }\DataTypeTok{newdata =}\NormalTok{ dt.test)}
\NormalTok{rmse_svr_}\DecValTok{13}\NormalTok{ <-}\StringTok{ }\KeywordTok{sqrt}\NormalTok{(}\KeywordTok{MSE}\NormalTok{(}\DataTypeTok{y_pred =}\NormalTok{ svr_pred_}\DecValTok{13}\NormalTok{, }\DataTypeTok{y_true =}\NormalTok{ dt.test}\OperatorTok{$}\NormalTok{ln_median_house_value))}
\NormalTok{nu_reg[}\DecValTok{13}\NormalTok{] <-}\StringTok{ }\NormalTok{rmse_svr_}\DecValTok{13}
\NormalTok{rmse_svr_}\DecValTok{13}
\end{Highlighting}
\end{Shaded}

\begin{verbatim}
## [1] 0.4395939
\end{verbatim}

\begin{Shaded}
\begin{Highlighting}[]
\NormalTok{svr_}\DecValTok{14}\NormalTok{ <-}\StringTok{ }\KeywordTok{svm}\NormalTok{(ln_median_house_value }\OperatorTok{~}\NormalTok{. , }
                 \DataTypeTok{data =}\NormalTok{ dt.train, }
                 \DataTypeTok{type =} \StringTok{"nu-regression"}\NormalTok{, }
                 \DataTypeTok{kernel =} \StringTok{"polynomial"}\NormalTok{, }\DataTypeTok{cost =} \FloatTok{0.01}\NormalTok{, }\DataTypeTok{gamma =} \FloatTok{0.1}\NormalTok{, }\DataTypeTok{degree =} \DecValTok{3}\NormalTok{)}
\NormalTok{svr_pred_}\DecValTok{14}\NormalTok{ <-}\StringTok{ }\KeywordTok{predict}\NormalTok{(svr_}\DecValTok{5}\NormalTok{, }\DataTypeTok{newdata =}\NormalTok{ dt.test)}
\NormalTok{rmse_svr_}\DecValTok{14}\NormalTok{ <-}\StringTok{ }\KeywordTok{sqrt}\NormalTok{(}\KeywordTok{MSE}\NormalTok{(}\DataTypeTok{y_pred =}\NormalTok{ svr_pred_}\DecValTok{14}\NormalTok{, }\DataTypeTok{y_true =}\NormalTok{ dt.test}\OperatorTok{$}\NormalTok{ln_median_house_value))}
\NormalTok{nu_reg[}\DecValTok{14}\NormalTok{] <-}\StringTok{ }\NormalTok{rmse_svr_}\DecValTok{14}
\NormalTok{rmse_svr_}\DecValTok{14}
\end{Highlighting}
\end{Shaded}

\begin{verbatim}
## [1] 0.4395939
\end{verbatim}

\begin{Shaded}
\begin{Highlighting}[]
\NormalTok{svr_}\DecValTok{15}\NormalTok{ <-}\StringTok{ }\KeywordTok{svm}\NormalTok{(ln_median_house_value }\OperatorTok{~}\NormalTok{. , }
                 \DataTypeTok{data =}\NormalTok{ dt.train, }
                 \DataTypeTok{type =} \StringTok{"nu-regression"}\NormalTok{, }
                 \DataTypeTok{kernel =} \StringTok{"polynomial"}\NormalTok{, }\DataTypeTok{cost =} \FloatTok{0.1}\NormalTok{, }\DataTypeTok{gamma =} \FloatTok{0.1}\NormalTok{, }\DataTypeTok{degree =} \DecValTok{3}\NormalTok{)}
\NormalTok{svr_pred_}\DecValTok{15}\NormalTok{ <-}\StringTok{ }\KeywordTok{predict}\NormalTok{(svr_}\DecValTok{5}\NormalTok{, }\DataTypeTok{newdata =}\NormalTok{ dt.test)}
\NormalTok{rmse_svr_}\DecValTok{15}\NormalTok{ <-}\StringTok{ }\KeywordTok{sqrt}\NormalTok{(}\KeywordTok{MSE}\NormalTok{(}\DataTypeTok{y_pred =}\NormalTok{ svr_pred_}\DecValTok{15}\NormalTok{, }\DataTypeTok{y_true =}\NormalTok{ dt.test}\OperatorTok{$}\NormalTok{ln_median_house_value))}
\NormalTok{nu_reg[}\DecValTok{15}\NormalTok{] <-}\StringTok{ }\NormalTok{rmse_svr_}\DecValTok{15}
\NormalTok{rmse_svr_}\DecValTok{15}
\end{Highlighting}
\end{Shaded}

\begin{verbatim}
## [1] 0.4395939
\end{verbatim}

\begin{Shaded}
\begin{Highlighting}[]
\NormalTok{svr_}\DecValTok{16}\NormalTok{ <-}\StringTok{ }\KeywordTok{svm}\NormalTok{(ln_median_house_value }\OperatorTok{~}\NormalTok{. , }
                 \DataTypeTok{data =}\NormalTok{ dt.train, }
                 \DataTypeTok{type =} \StringTok{"nu-regression"}\NormalTok{, }
                 \DataTypeTok{kernel =} \StringTok{"polynomial"}\NormalTok{, }\DataTypeTok{cost =} \DecValTok{1}\NormalTok{, }\DataTypeTok{gamma =} \FloatTok{0.1}\NormalTok{, }\DataTypeTok{degree =} \DecValTok{3}\NormalTok{)}
\NormalTok{svr_pred_}\DecValTok{16}\NormalTok{ <-}\StringTok{ }\KeywordTok{predict}\NormalTok{(svr_}\DecValTok{5}\NormalTok{, }\DataTypeTok{newdata =}\NormalTok{ dt.test)}
\NormalTok{rmse_svr_}\DecValTok{16}\NormalTok{ <-}\StringTok{ }\KeywordTok{sqrt}\NormalTok{(}\KeywordTok{MSE}\NormalTok{(}\DataTypeTok{y_pred =}\NormalTok{ svr_pred_}\DecValTok{13}\NormalTok{, }\DataTypeTok{y_true =}\NormalTok{ dt.test}\OperatorTok{$}\NormalTok{ln_median_house_value))}
\NormalTok{nu_reg[}\DecValTok{16}\NormalTok{] <-}\StringTok{ }\NormalTok{rmse_svr_}\DecValTok{16}
\NormalTok{rmse_svr_}\DecValTok{16}
\end{Highlighting}
\end{Shaded}

\begin{verbatim}
## [1] 0.4395939
\end{verbatim}

Vemos el vector que contiene los errores de cada modelo de las Nu-SVRs.

\begin{Shaded}
\begin{Highlighting}[]
\NormalTok{nu_reg}
\end{Highlighting}
\end{Shaded}

\begin{verbatim}
##  [1] 0.3675666 0.3551530 0.3561244 0.3564781 0.4395939 0.4395939 0.4395939
##  [8] 0.4395939 0.4395939 0.4395939 0.4395939 0.4395939 0.4395939 0.4395939
## [15] 0.4395939 0.4395939
\end{verbatim}

De igual forma que en el apartado anterior con los Epsilon-SVR, vamos a
visualizar los coeficientes de solamente cuatro modelos.

\begin{Shaded}
\begin{Highlighting}[]
\KeywordTok{coef}\NormalTok{(svr_}\DecValTok{2}\NormalTok{)}
\end{Highlighting}
\end{Shaded}

\begin{verbatim}
##                  (Intercept)                median_income 
##                 -0.004499637                  1.169082036 
##              median_income_2              median_income_3 
##                  0.145689816                 -0.426531659 
##        ln_housing_median_age    ln_total_rooms_population 
##                  0.170291604                 -0.578878811 
## ln_total_bedrooms_population     ln_population_households 
##                  0.464685556                 -0.210966960 
##                ln_households 
##                  0.080924952
\end{verbatim}

\begin{Shaded}
\begin{Highlighting}[]
\KeywordTok{t}\NormalTok{(svr_}\DecValTok{5}\OperatorTok{$}\NormalTok{coefs) }\OperatorTok\StringTok{ }\NormalTok{svr_}\DecValTok{5}\OperatorTok{$}\NormalTok{SV}
\end{Highlighting}
\end{Shaded}

\begin{verbatim}
##      median_income median_income_2 median_income_3 ln_housing_median_age
## [1,]      4.548788        4.134815        3.575148             0.7963854
##      ln_total_rooms_population ln_total_bedrooms_population
## [1,]                  1.431569                    0.7846614
##      ln_population_households ln_households
## [1,]                -1.570198      1.058001
\end{verbatim}

\begin{Shaded}
\begin{Highlighting}[]
\NormalTok{svr_}\DecValTok{5}\OperatorTok{$}\NormalTok{rho}
\end{Highlighting}
\end{Shaded}

\begin{verbatim}
## [1] -0.1878867
\end{verbatim}

\begin{Shaded}
\begin{Highlighting}[]
\KeywordTok{t}\NormalTok{(svr_}\DecValTok{9}\OperatorTok{$}\NormalTok{coefs) }\OperatorTok\StringTok{ }\NormalTok{svr_}\DecValTok{9}\OperatorTok{$}\NormalTok{SV}
\end{Highlighting}
\end{Shaded}

\begin{verbatim}
##      median_income median_income_2 median_income_3 ln_housing_median_age
## [1,]      5.250779        4.132941        2.862027             0.5703704
##      ln_total_rooms_population ln_total_bedrooms_population
## [1,]                  2.747747                       1.5572
##      ln_population_households ln_households
## [1,]                -2.292135      1.276058
\end{verbatim}

\begin{Shaded}
\begin{Highlighting}[]
\NormalTok{svr_}\DecValTok{9}\OperatorTok{$}\NormalTok{rho}
\end{Highlighting}
\end{Shaded}

\begin{verbatim}
## [1] 0.05810638
\end{verbatim}

\begin{Shaded}
\begin{Highlighting}[]
\KeywordTok{t}\NormalTok{(svr_}\DecValTok{13}\OperatorTok{$}\NormalTok{coefs) }\OperatorTok\StringTok{ }\NormalTok{svr_}\DecValTok{13}\OperatorTok{$}\NormalTok{SV}
\end{Highlighting}
\end{Shaded}

\begin{verbatim}
##      median_income median_income_2 median_income_3 ln_housing_median_age
## [1,]      4.558325        3.538595        2.404104             0.8691997
##      ln_total_rooms_population ln_total_bedrooms_population
## [1,]                  1.994853                      1.17753
##      ln_population_households ln_households
## [1,]                -1.953529     0.9341405
\end{verbatim}

\begin{Shaded}
\begin{Highlighting}[]
\NormalTok{svr_}\DecValTok{13}\OperatorTok{$}\NormalTok{rho}
\end{Highlighting}
\end{Shaded}

\begin{verbatim}
## [1] 0.02083598
\end{verbatim}

La influencia de las variables en cada modelo es exactamente igual que
el caso de los modelos de las Epsilon-SVRs.

\hypertarget{otros-modelos-de-regresiuxf3n}{%
\section{Otros modelos de
Regresión}\label{otros-modelos-de-regresiuxf3n}}

Aplicaremos otros algoritmos de regresión a nuestro conjunto de datos
para comparar los resultados con los modelos de SVR.

\hypertarget{regresiuxf3n-lineal-muxfaltiple}{%
\subsection{Regresión lineal
múltiple}\label{regresiuxf3n-lineal-muxfaltiple}}

Recordamos que la regresión lineal múltiple es un método en el cual
predecimos una variable dependiente o respuesta a partir de una
combinación lineal de las variables independientes o explicativas.

Entrenamos el modelo de regresión lineal múltiple con nuestro conjunto
de entrenamiento.

\begin{Shaded}
\begin{Highlighting}[]
\NormalTok{reg_lineal <-}\StringTok{ }\KeywordTok{lm}\NormalTok{(ln_median_house_value }\OperatorTok{~}\NormalTok{.,}
               \DataTypeTok{data =}\NormalTok{ dt.train)}
\KeywordTok{summary}\NormalTok{(reg_lineal)}
\end{Highlighting}
\end{Shaded}

\begin{verbatim}
## 
## Call:
## lm(formula = ln_median_house_value ~ ., data = dt.train)
## 
## Residuals:
##      Min       1Q   Median       3Q      Max 
## -3.15125 -0.21334  0.00129  0.20587  2.65018 
## 
## Coefficients:
##                              Estimate Std. Error t value Pr(>|t|)    
## (Intercept)                  10.96562    0.16348  67.077  < 2e-16 ***
## median_income                 5.29216    0.20255  26.128  < 2e-16 ***
## median_income_2              -1.01337    0.52020  -1.948   0.0514 .  
## median_income_3              -1.71200    0.38404  -4.458 8.34e-06 ***
## ln_housing_median_age         0.63618    0.02219  28.671  < 2e-16 ***
## ln_total_rooms_population    -8.59352    0.18355 -46.819  < 2e-16 ***
## ln_total_bedrooms_population  8.27702    0.25394  32.595  < 2e-16 ***
## ln_population_households     -2.81372    0.17520 -16.060  < 2e-16 ***
## ln_households                 0.36374    0.03314  10.977  < 2e-16 ***
## ---
## Signif. codes:  0 '***' 0.001 '**' 0.01 '*' 0.05 '.' 0.1 ' ' 1
## 
## Residual standard error: 0.3571 on 14560 degrees of freedom
## Multiple R-squared:  0.6059, Adjusted R-squared:  0.6057 
## F-statistic:  2798 on 8 and 14560 DF,  p-value: < 2.2e-16
\end{verbatim}

Vemos que las variables que influyen de forma positiva sobre el precio
medio de las viviendas son: \texttt{median\_income},
\texttt{ln\_housing\_median\_age},
\texttt{ln\_total\_bedrooms\_population} y \texttt{ln\_households}; las
otras variables influyen de forma negativa.

A modo de ejemplo comentamos qué significa el coeficiente de la variable
\texttt{median\_income}. El valor del coeficiente que nos da el modelo
es 5.29216 y lo interpretamos como: un aumento de una unidad en
\texttt{median\_income} hace que la variable
\texttt{ln\_median\_house\_value} suba 5.29216 unidades. Vemos que la
variable \texttt{median\_income\_2} es la única que no es significativa
a un nivel de significación del 5\%.

Con el conjunto de validación crearemos las predicciones con el modelo
creado anteriormente y después calcularemos la raíz del error cuadrático
medio (RMSE) y la utilizaremos como medida de precisión de nuestro
modelo.

\begin{Shaded}
\begin{Highlighting}[]
\NormalTok{reg_lineal_pred <-}\StringTok{ }\KeywordTok{predict}\NormalTok{(reg_lineal, }\DataTypeTok{newdata =}\NormalTok{ dt.test)}
\NormalTok{rmse_reg_lin <-}\StringTok{ }\KeywordTok{sqrt}\NormalTok{(}\KeywordTok{MSE}\NormalTok{(}\DataTypeTok{y_pred =}\NormalTok{ reg_lineal_pred, }\DataTypeTok{y_true =}\NormalTok{ dt.test}\OperatorTok{$}\NormalTok{ln_median_house_value))}
\NormalTok{rmse_reg_lin}
\end{Highlighting}
\end{Shaded}

\begin{verbatim}
## [1] 0.3554993
\end{verbatim}

Nos sale un error muy parecido al mínimo error que tenemos de los
modelos SVR.

\hypertarget{knn-k-nearest-neighbors}{%
\subsection{KNN (K-nearest Neighbors)}\label{knn-k-nearest-neighbors}}

El método de K-Nearest Neighbors es otro método de regresión. El
algoritmo reconoce patrones en los datos sin un aprendizaje específico,
consiguiendo un criterio de agrupamiento de los datos a partir de un
conjunto de entrenamiento.

A continuación vamos a predecir nuestras observaciones, para más tarde
calcular la raíz del error cuadrático medio.

\begin{Shaded}
\begin{Highlighting}[]
\KeywordTok{set.seed}\NormalTok{(}\DecValTok{248}\NormalTok{)}
\NormalTok{dt_knn <-}\StringTok{ }\NormalTok{FNN}\OperatorTok{::}\KeywordTok{knn.reg}\NormalTok{(}\DataTypeTok{train =}\NormalTok{ dt.train[,}\OperatorTok{-}\DecValTok{9}\NormalTok{], }\DataTypeTok{test =}\NormalTok{ dt.test[,}\OperatorTok{-}\DecValTok{9}\NormalTok{], }\DataTypeTok{y =}\NormalTok{ dt.train}\OperatorTok{$}\NormalTok{ln_median_house_value, }\DataTypeTok{k =} \DecValTok{10}\NormalTok{)}
\NormalTok{rmse_knn <-}\StringTok{ }\KeywordTok{sqrt}\NormalTok{(}\KeywordTok{MSE}\NormalTok{(}\DataTypeTok{y_pred =}\NormalTok{ dt_knn}\OperatorTok{$}\NormalTok{pred, }\DataTypeTok{y_true =}\NormalTok{ dt.test}\OperatorTok{$}\NormalTok{ln_median_house_value))}
\NormalTok{rmse_knn}
\end{Highlighting}
\end{Shaded}

\begin{verbatim}
## [1] 0.351206
\end{verbatim}

El error que conseguimos es 0.351206, muy parecido a los otros que hemos
conseguido.

\hypertarget{regression-tree}{%
\subsection{Regression Tree}\label{regression-tree}}

Vamos a realizar un modelo basado en un árbol de regresión. Utilizaremos
nuestro conjunto de entrenamiento para entrenar los datos, para más
tarde predecir y calcular la raíz del error cuadrático medio.

\begin{Shaded}
\begin{Highlighting}[]
\NormalTok{reg_tree <-}\StringTok{ }\KeywordTok{rpart}\NormalTok{(ln_median_house_value }\OperatorTok{~}\NormalTok{., }\DataTypeTok{data =}\NormalTok{ dt.train, }\DataTypeTok{method  =} \StringTok{"anova"}\NormalTok{)}
\NormalTok{tree_prediction <-}\StringTok{ }\KeywordTok{predict}\NormalTok{(reg_tree, }\DataTypeTok{newdata =}\NormalTok{ dt.test) }
\NormalTok{rmse_tree <-}\StringTok{ }\KeywordTok{sqrt}\NormalTok{(}\KeywordTok{MSE}\NormalTok{(}\DataTypeTok{y_pred =}\NormalTok{ tree_prediction, }\DataTypeTok{y_true =}\NormalTok{ dt.test}\OperatorTok{$}\NormalTok{ln_median_house_value))}
\NormalTok{rmse_tree}
\end{Highlighting}
\end{Shaded}

\begin{verbatim}
## [1] 0.4084926
\end{verbatim}

El error es bastante más grande que el resto, probablemente debido a que
es un modelo muy simple.

\begin{Shaded}
\begin{Highlighting}[]
\KeywordTok{rpart.plot}\NormalTok{(reg_tree)}
\end{Highlighting}
\end{Shaded}

\includegraphics{SVM_regresion_files/figure-latex/unnamed-chunk-73-1.pdf}

Con este plot vemos cómo nos ha quedado el árbol de regresión.

\hypertarget{random-forest}{%
\subsection{Random Forest}\label{random-forest}}

Un solo árbol de regresión no es suficiente, ya que sufren de problemas
de sesgo y varianza en las predicciones. Para poder mejorar tanto los
problemas comentados y una mejor precisión, vamos a introducir el
concepto de Random Forest. Random Forest es un método tipo ensemble que
está formado por un grupo de modelos predictivos alcanzando una
precisión y una estabilidad mejores. Los árboles sufren problemas de
sesgo y varianza en las predicciones; Random Forest forma parte de los
métodos de Bagging y éstos funcionan de la siguiente manera:

\begin{itemize}
\tightlist
\item
  Crear muchos subconjuntos de datos.
\item
  Construir múltiples modelos.
\item
  Combinar los modelos construidos.
\end{itemize}

Random Forest crea un grupo de modelos aparentemente débiles (múltiples
árboles de decisión), para combinarlos y transformarlos en un modelo más
potente.

\begin{Shaded}
\begin{Highlighting}[]
\NormalTok{rf <-}\StringTok{ }\KeywordTok{randomForest}\NormalTok{(ln_median_house_value}\OperatorTok{~}\NormalTok{. , }\DataTypeTok{data =}\NormalTok{ dt.train, }\DataTypeTok{ntree =} \DecValTok{1000}\NormalTok{)}
\NormalTok{rf_pred <-}\StringTok{ }\KeywordTok{predict}\NormalTok{(rf, }\DataTypeTok{newdata =}\NormalTok{ dt.test)}
\NormalTok{rmse_rf <-}\StringTok{ }\KeywordTok{sqrt}\NormalTok{(}\KeywordTok{MSE}\NormalTok{(}\DataTypeTok{y_pred =}\NormalTok{ rf_pred, }\DataTypeTok{y_true =}\NormalTok{ dt.test}\OperatorTok{$}\NormalTok{ln_median_house_value))}
\NormalTok{rmse_rf}
\end{Highlighting}
\end{Shaded}

\begin{verbatim}
## [1] 0.3315596
\end{verbatim}

Con este algoritmo más sofisticado conseguimos el mejor error obtenido
hasta el momento: 0.3315596.

\hypertarget{ann}{%
\subsection{ANN}\label{ann}}

Para construir nuestras redes neuronales artificiales, hemos considerado
los siguientes valores de los parámetros:

\begin{itemize}
\item
  threshold = 1
\item
  err.fct = ``sse''
\item
  linear.output = TRUE, ya que se trata de regresión
\item
  learningrate = 0.1
\item
  act.fct = ``logistic''
\end{itemize}

El primero de los dos modelos que ejecutamos es con una capa de 3
neuronas.

\begin{Shaded}
\begin{Highlighting}[]
\KeywordTok{set.seed}\NormalTok{(}\DecValTok{284}\NormalTok{)}
\NormalTok{ANN <-}\StringTok{ }\KeywordTok{neuralnet}\NormalTok{(ln_median_house_value}\OperatorTok{~}\NormalTok{. , }\DataTypeTok{data =}\NormalTok{ dt.train, }
                 \DataTypeTok{hidden =} \KeywordTok{c}\NormalTok{(}\DecValTok{3}\NormalTok{),}
                 \DataTypeTok{threshold =} \DecValTok{1}\NormalTok{,}
                 \DataTypeTok{err.fct=}\StringTok{"sse"}\NormalTok{,}
                 \DataTypeTok{linear.output=}\OtherTok{TRUE}\NormalTok{,}
                 \DataTypeTok{learningrate =} \FloatTok{0.1}\NormalTok{,}
                 \DataTypeTok{act.fct =} \StringTok{"logistic"}\NormalTok{)}

\NormalTok{ANN_pred <-}\StringTok{ }\KeywordTok{compute}\NormalTok{(ANN, dt.test[, }\DecValTok{-9}\NormalTok{])}\OperatorTok{$}\NormalTok{net.result}
\NormalTok{rmse_ann_}\DecValTok{1}\NormalTok{ <-}\StringTok{ }\KeywordTok{sqrt}\NormalTok{(}\KeywordTok{MSE}\NormalTok{(}\DataTypeTok{y_pred =}\NormalTok{ ANN_pred, }\DataTypeTok{y_true =}\NormalTok{ dt.test}\OperatorTok{$}\NormalTok{ln_median_house_value))}
\NormalTok{rmse_ann_}\DecValTok{1}
\end{Highlighting}
\end{Shaded}

\begin{verbatim}
## [1] 0.341986
\end{verbatim}

\begin{Shaded}
\begin{Highlighting}[]
\KeywordTok{plot}\NormalTok{(ANN)}
\end{Highlighting}
\end{Shaded}

El segundo es con 3 capas de 5, 2 y 2 neurones respectivamente.

\begin{Shaded}
\begin{Highlighting}[]
\KeywordTok{set.seed}\NormalTok{(}\DecValTok{284}\NormalTok{)}
\NormalTok{ANN <-}\StringTok{ }\KeywordTok{neuralnet}\NormalTok{(ln_median_house_value}\OperatorTok{~}\NormalTok{. , }\DataTypeTok{data =}\NormalTok{ dt.train, }
                 \DataTypeTok{hidden =} \KeywordTok{c}\NormalTok{(}\DecValTok{5}\NormalTok{,}\DecValTok{2}\NormalTok{,}\DecValTok{2}\NormalTok{),}
                 \DataTypeTok{threshold =} \DecValTok{1}\NormalTok{,}
                 \DataTypeTok{err.fct=}\StringTok{"sse"}\NormalTok{,}
                 \DataTypeTok{linear.output=}\OtherTok{TRUE}\NormalTok{,}
                 \DataTypeTok{learningrate =} \FloatTok{0.1}\NormalTok{,}
                 \DataTypeTok{act.fct =} \StringTok{"logistic"}\NormalTok{)}

\NormalTok{ANN_pred <-}\StringTok{ }\KeywordTok{compute}\NormalTok{(ANN, dt.test[, }\DecValTok{-9}\NormalTok{])}\OperatorTok{$}\NormalTok{net.result}
\NormalTok{rmse_ann_}\DecValTok{2}\NormalTok{ <-}\StringTok{ }\KeywordTok{sqrt}\NormalTok{(}\KeywordTok{MSE}\NormalTok{(}\DataTypeTok{y_pred =}\NormalTok{ ANN_pred, }\DataTypeTok{y_true =}\NormalTok{ dt.test}\OperatorTok{$}\NormalTok{ln_median_house_value))}
\NormalTok{rmse_ann_}\DecValTok{2}
\end{Highlighting}
\end{Shaded}

\begin{verbatim}
## [1] 0.3338819
\end{verbatim}

\begin{Shaded}
\begin{Highlighting}[]
\KeywordTok{plot}\NormalTok{(ANN)}
\end{Highlighting}
\end{Shaded}

De los dos modelos, el que tiene menor error es el segundo con 3 capas
de 5, 2 y 2 neuronas respectivamente.

\hypertarget{conclusiuxf3n}{%
\section{Conclusión}\label{conclusiuxf3n}}

Reunimos todos los errores que obtenemos en cada uno de los algoritmos
de regresión. En el caso de los SVR, cogemos el mejor de los modelos de
Epsilon-SVR y, por el otro lado, el mejor de los de Nu-SVR. Los
visualizamos en un tabla:

\begin{Shaded}
\begin{Highlighting}[]
\NormalTok{resultados <-}\StringTok{ }\KeywordTok{data.frame}\NormalTok{(}\StringTok{"RMSE"}\NormalTok{ =}\StringTok{ }\KeywordTok{c}\NormalTok{(eps_reg[}\DecValTok{2}\NormalTok{], nu_reg[}\DecValTok{2}\NormalTok{], rmse_reg_lin, rmse_knn, rmse_tree, rmse_rf,}
\NormalTok{                                    rmse_ann_}\DecValTok{2}\NormalTok{))}

\KeywordTok{rownames}\NormalTok{(resultados) <-}\StringTok{ }\KeywordTok{c}\NormalTok{(}\StringTok{"Epsilon-SVR"}\NormalTok{, }\StringTok{"Nu-SVR"}\NormalTok{, }\StringTok{"Regresión lineal múltiple", "}\NormalTok{K}\OperatorTok{-}\NormalTok{NN}\StringTok{", "}\NormalTok{Regression tree}\StringTok{",}
\StringTok{                           "}\NormalTok{Random}\OperatorTok{-}\NormalTok{Forest}\StringTok{", "}\NormalTok{ANN}\StringTok{")}
\StringTok{resultados}
\end{Highlighting}
\end{Shaded}

\begin{verbatim}
##                                RMSE
## Epsilon-SVR               0.3553593
## Nu-SVR                    0.3551530
## Regresión lineal múltiple 0.3554993
## K-NN                      0.3512060
## Regression tree           0.4084926
## Random-Forest             0.3315596
## ANN                       0.3338819
\end{verbatim}

Los dos mejores algoritmos que han cosechado un menor error en la
predicción son Random-Forest (probablemente por ser un algoritmo mucho
más sofisticado) y ANN. Realmente los modelos de SVR solamente han
variado en el caso de kernel lineal, en los otros tipos de kernel el
resultado se mantenía constante, cosa que nos hace sospechar de que esos
modelos no están trabajando de la forma óptima que deberían. Recalcamos
que tanto K-NN como la regresión lineal múltiple son métodos simples que
han alcanzado un error similar que los modelos de SVR, siendo estos
últimos unos algoritmos muchos más pesados y potentes.

En definitiva, si nos ceñimos por el error de predicción los modelos que
seleccionamos son el Random-Forest y el ANN con 3 capas de 5, 2 y 2
neuronas respectivamente.

\end{document}
